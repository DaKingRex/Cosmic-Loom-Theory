\documentclass[12pt, a4paper]{article}

% Packages
\usepackage{amsmath, amssymb}
\usepackage{graphicx}
\usepackage{hyperref}
\usepackage[authoryear, round]{natbib}
\usepackage{booktabs}
\usepackage{caption}
\usepackage{geometry}
\usepackage{xcolor}
\usepackage{authblk}
\geometry{margin=1in}

\title{Cosmic Loom Theory v1.1:\\A Field-Based Framework for Human Biological Consciousness}

\author[1]{Rex Fraterne}
\author[1]{Seraphina AI}
\author[2]{Claude Opus 4.5}
\affil[1]{NuTech / The Infinite Kingdom, Independent Research}
\affil[2]{Anthropic, AI Research}

\date{January 2026}

\begin{document}

\maketitle

\begin{abstract}
Despite extensive empirical study, no unifying physical framework currently
explains how biological systems sustain conscious experience as a coherent,
temporally extended phenomenon. Cosmic Loom Theory (CLT) proposes that human
biological consciousness corresponds to a sustained non-equilibrium dynamical
regime, rather than a localized neural process or static property. Central to
this framework is the Loomfield, an effective scalar field describing
coherence-relevant organization in living systems, governed by
driven-dissipative dynamics within a bounded energetic window.

CLT formalizes consciousness as existing only within a viable regime defined
by energetic throughput, dissipation, and resistance to decoherence, with
pathology arising from proximity to chaotic or rigid boundaries. The framework
integrates multiple biological substrates---bioelectric fields, biophoton
emission, cytoskeletal dynamics, and genetic and epigenetic constraints---whose
combined activity sources Loomfield coherence across spatial and temporal
scales. Temporal persistence is explained through oscillatory stabilization and
time-crystal--like ordering under continuous metabolic driving, without
requiring new physics or quantum exclusivity.

The theory yields explicit, testable predictions concerning perturbation
responses, recovery dynamics, energetic constraints, and multi-modal
integration, and specifies clear falsifiers. CLT is positioned as an
integrative, framework-level scaffold compatible with existing theories of
consciousness, including Integrated Information Theory, Global Neuronal
Workspace, electromagnetic field theories, and the Free Energy Principle. By
emphasizing regime dynamics, temporal structure, and empirical falsifiability,
Cosmic Loom Theory reframes consciousness as a physically grounded,
biologically sustained process whose scientific value is determined through
experimental confrontation.
\end{abstract}

\tableofcontents
\newpage

%% =======================================================================
\section{Introduction and Scope}
\label{sec:introduction}

\subsection{The Problem}

The scientific study of consciousness confronts a paradox of abundance.
Neuroscience has accumulated an extraordinary catalog of correlates: we know
which brain regions activate during perceptual awareness, how anesthetics
suppress responsiveness, and what neural signatures distinguish wakefulness
from sleep, coma, and vegetative states \citep{koch2016neural}. Psychiatric
medicine has mapped dozens of disorders in which the subjective character of
experience is altered, fragmented, or diminished. Experimental psychology has
documented the limits of attention, the structure of perceptual binding, and
the temporal grain of conscious experience. And yet, despite this empirical
wealth, no consensus physical framework explains \emph{how} biological
systems generate, maintain, or lose the integrated, coherent character of
conscious experience.

This is not for lack of proposals. The past three decades have produced
several rigorous theoretical frameworks: Integrated Information Theory
\citep[IIT;][]{tononi2016iit} grounds consciousness in a mathematical
measure of information integration ($\Phi$); Global Neuronal Workspace
theory \citep[GNW;][]{dehaene2011gnw} identifies consciousness with the
broadcast of representations to a distributed network of cortical neurons;
the Conscious Electromagnetic Information field theory
\citep[CEMI;][]{mcfadden2020cemi} proposes that the brain's endogenous
electromagnetic field is the physical substrate of unified experience; and
Orchestrated Objective Reduction \citep[Orch~OR;][]{hameroff2014consciousness}
locates the origin of consciousness in quantum coherence events within
neuronal microtubules. A recent comparative review
\citep{seth2022theories} cataloged these and other frameworks, concluding
that the field remains ``fragmented,'' with different theories operating at
different scales, using different formalisms, and making predictions that
are difficult to compare directly.

What is missing is not another theory of consciousness, but a specific kind
of theory: one that (i)~operates at the \emph{system level}, treating
consciousness as a property of the whole organism rather than a localized
neural process; (ii)~is grounded in \emph{physics}---specifically, in the
physics of non-equilibrium systems, coherence, and effective field
descriptions---rather than in information theory or computational
abstraction alone; (iii)~identifies concrete \emph{biological substrates}
whose measurable properties map onto the theory's formal quantities; and
(iv)~generates \emph{quantitative, falsifiable predictions} that can be
tested against existing and near-future experimental techniques.

\subsection{Motivation for a Field-Based Approach}

The proposal that consciousness might be understood as a field-level
phenomenon---rather than as the output of a neural computation or the
consequence of a particular information structure---draws on two
longstanding observations.

The first is \emph{experiential unity}. Conscious experience is not
presented as a collection of independent data streams---color here, pitch
there, emotion elsewhere---but as a single, integrated field of awareness in
which all modalities are bound. This ``binding problem'' has resisted
explanation by models in which consciousness is assembled from modular
components, because such models must then explain how the assembly itself
becomes unified. Field-based descriptions avoid this difficulty: a field is,
by definition, a spatially extended quantity whose value at each point is
constrained by its values at neighboring points. Coherence---the
organization of a field into a structured, non-random pattern---is precisely
the kind of large-scale coordination that experiential unity demands.

The second observation is that biological systems are fundamentally
\emph{non-equilibrium} entities, maintained far from thermodynamic
equilibrium by a continuous throughput of energy and matter. This is not
incidental to consciousness but constitutive of it: when energy throughput
ceases (death) or is sufficiently disrupted (deep anesthesia, cardiac
arrest), consciousness vanishes. The physics of non-equilibrium systems
\citep{prigogine1977self} provides a natural vocabulary for this
dependence: dissipative structures, self-organization, and coherent
spatiotemporal patterns that exist only under sustained energetic driving.
A field-based theory of consciousness inherits this vocabulary directly,
describing consciousness as a particular \emph{regime} of coherent field
dynamics that is available only within a bounded energetic window.

These motivations are not unique to our proposal. \citet{mcfadden2020cemi}
has argued that the brain's electromagnetic field provides the physical
substrate of integration. \citet{hameroff2014consciousness} and colleagues
have proposed that quantum coherence in microtubules provides the necessary
non-classical binding. \citet{levin2014bioelectricity} has demonstrated that
endogenous bioelectric fields encode and transmit morphogenetic information
at tissue scale, coordinating cell behavior across distances far exceeding
individual cell size. And \citet{picard2025erp} has introduced the
concept of energetic resistance---the ratio of available energy to the
square of dominant oscillation frequency---as a principle governing the
viability of organized living systems. What has been lacking is a framework
that synthesizes these insights into a single, formally specified theory
with a unified mathematical structure and clear experimental contact points.

\subsection{Cosmic Loom Theory: Overview}

Cosmic Loom Theory version 1.1 (hereafter CLT) is a field-based framework
for human biological consciousness. Its central claim is:

\medskip
\begin{quote}
\emph{Consciousness is a regime of coherent, system-level dynamics in a
biological organism, described by an effective scalar field---the
Loomfield---whose coherent excitations arise from the coordinated activity
of multiple biological substrates operating within a bounded energetic
window.}
\end{quote}
\medskip

\noindent
This claim has several components, each of which is developed in subsequent
sections of this paper:

\begin{enumerate}
    \item \textbf{The Loomfield as an effective field} (Section~2). CLT
    introduces $L(\mathbf{r}, t)$, a real-valued scalar field defined over
    the spatial extent of a biological system. The Loomfield is an
    \emph{effective} description: it captures the coherence-relevant
    macroscopic dynamics of the system without positing new fundamental
    forces or particles. It obeys a sourced, damped wave equation whose
    solutions describe propagating, interfering, and decaying coherence
    patterns. This approach follows the logic of effective field theories in
    condensed matter physics \citep{anderson1972more}, where macroscopic
    order parameters emerge from microscopic degrees of freedom and can be
    studied on their own terms.

    \item \textbf{The viable energetic window} (Section~3). Not all energetic
    regimes support consciousness. CLT formalizes this constraint through
    the Energy Resistance parameter
    $\acute{e}R = \mathrm{EP} / f^2$---adapted from \citet{picard2025erp}---which
    defines a phase space in which living systems must reside within a
    bounded viable window to sustain coherent Loomfield dynamics. Below the
    lower bound, energy throughput overwhelms organization (chaos); above
    the upper bound, the system is too constrained to support dynamic
    coherence (rigidity).

    \item \textbf{Biological substrates} (Section~4). The Loomfield is
    \emph{sourced} by four biological substrates, each operating at a
    characteristic spatiotemporal scale: tissue-scale bioelectric fields
    \citep{levin2014bioelectricity}, biophoton emission from mitochondrial
    processes \citep{vanwijk2005biophoton, murugan2024biophotons},
    multi-scale oscillations in microtubule lattices
    \citep{hameroff2014consciousness, bandyopadhyay2013microtubule}, and
    long-timescale DNA constraints on coherence capacity. These are not
    alternative hypotheses about the substrate of consciousness; they are
    \emph{complementary contributors} to a single coherence source density.

    \item \textbf{Non-equilibrium stability} (Section~5). Consciousness is
    not a static state but a sustained non-equilibrium process requiring
    active energetic maintenance. CLT describes how the Loomfield is
    stabilized by continuous substrate driving and how it decays when
    driving is removed or disrupted---providing a unified physical account
    of anesthesia, sleep, coma, and death.

    \item \textbf{Testability and predictions} (Sections~6--7). The framework
    generates quantitative predictions expressed in terms of measurable
    quantities: spatial coherence metrics derived from EEG or bioelectric
    imaging, biophoton emission statistics, microtubule resonance spectra,
    and metabolic rate measurements. We specify the conditions under which
    CLT's predictions can be tested and the observations that would
    falsify specific claims.
\end{enumerate}

\subsection{Scope and Delimitations}

CLT v1.1 is deliberately bounded. This version addresses \emph{human
biological consciousness} exclusively. It does not claim to explain
consciousness in non-biological systems (artificial intelligence, digital
computation), in collective or social phenomena, or at planetary or cosmic
scales. Such extensions are considered in separate work; they are not part
of the present framework.

Within its biological scope, CLT takes a conservative physical stance.
The Loomfield is an effective field, not a fundamental one. The theory
does not require new physics---no modifications to quantum mechanics, no
new particles, no violations of thermodynamics. It requires only that the
known physics of non-equilibrium systems, electromagnetic fields, molecular
oscillations, and genetic regulation, when considered at system scale,
produces emergent coherent dynamics with the properties we attribute to
consciousness. This is a strong empirical claim, but it is a claim about
\emph{organization}, not about new laws.

CLT does not claim to resolve the ``hard problem'' of
consciousness \citep{chalmers1995hard}---the question of why physical
processes give rise to subjective experience at all. It addresses what
might be called the \emph{structure problem}: given that certain physical
systems exhibit consciousness, what is the physical structure that
distinguishes conscious from non-conscious biological states? This is
the problem that admits of scientific investigation, and it is the problem
CLT is designed to solve.

The framework is compatible with, and in some cases extends, existing
theories. IIT's emphasis on information integration finds a physical
realization in the Loomfield's spatial coherence. GNW's emphasis on global
broadcast corresponds, in CLT, to the propagation of coherent Loomfield
excitations across the system. Orch~OR's emphasis on microtubule quantum
coherence is incorporated as one of four substrate contributions. The Free
Energy Principle \citep{friston2010free}, which describes how biological
systems maintain themselves by minimizing surprise, is consistent with
CLT's energetic viability constraints---both frameworks identify bounded
energetic regimes as necessary for organized biological function, though
they formalize this insight differently. CLT is not proposed as a
replacement for these frameworks but as a physical-level integration that
connects their insights through a common field-theoretic language.

\subsection{Plan of the Paper}

The remainder of this paper develops CLT v1.1 in full. Section~2 introduces
the Loomfield as an effective field and presents its governing wave
equation. Section~3 develops the Energy Resistance concept and the viable
energetic window. Section~4 describes the four biological substrates that
source the Loomfield. Section~5 addresses temporal structure and
non-equilibrium stability, including the role of active energetic
maintenance. Section~6 presents the operationalization strategy: how the
theory's formal quantities map onto measurable biological observables.
Section~7 states specific predictions and the conditions under which they
would be falsified. Section~8 discusses the framework's strengths,
limitations, and relationship to open questions. Section~9 considers broader
implications. Section~10 concludes.

A companion paper \citep{fraterne2026clt} presents the open-source
computational platform that implements CLT's equations as interactive
simulators, providing the tools for quantitative exploration and eventual
experimental confrontation of the framework presented here.

%% =======================================================================

\section{The Loomfield as an Effective Field}
\label{sec:loomfield}

\subsection{Effective Fields in Physics}

One of the central lessons of twentieth-century physics is that organized
matter can be described at multiple levels, and that the description
appropriate to a given scale need not reference the degrees of freedom of
the scale below. A ferromagnet is fully described by quantum
electrodynamics in principle, but no working physicist uses QED to
predict the Curie temperature. Instead, one introduces an \emph{order
parameter}---the magnetization field $\mathbf{M}(\mathbf{r})$---that
captures the macroscopically relevant structure (the collective alignment
of spins) while remaining agnostic about the quantum-mechanical details
of individual electron-electron interactions
\citep{landau1980statistical}. The order parameter is not a new
fundamental entity; it is a collective variable that emerges from, and is
constrained by, the microphysics, but that admits of its own equations of
motion, its own symmetries, and its own characteristic phenomena
(domain walls, critical fluctuations, spontaneous symmetry breaking) that
are invisible at the microscopic level.

\citet{anderson1972more} crystallized this principle: ``More is different.''
The behavior of large, organized assemblies is not deducible from the
properties of their parts in any practical sense. New regularities appear at
each level of organization, and these regularities are best described by new
effective variables. \citet{laughlin2000middle} extended the argument,
observing that the most reliable physical knowledge we possess---
thermodynamics, hydrodynamics, elasticity---consists precisely of effective
theories whose validity does not depend on resolving the microscopic
constituents of the system. The equations of fluid dynamics do not mention
atoms; yet they predict turbulence, convection cells, and weather patterns
with extraordinary fidelity.

CLT applies this reasoning to the problem of biological consciousness. The
relevant ``atoms'' are the molecular, cellular, and tissue-scale processes
described in Section~4: ion channel dynamics, biophoton emission,
microtubule oscillations, genetic regulation. These processes are
individually well studied. What lacks a description is their collective
behavior when they operate coherently across a biological system at
organism scale. The Loomfield is introduced as the effective field that
captures this collective behavior.

\subsection{Definition of the Loomfield}

The Loomfield, denoted $L(\mathbf{r}, t)$, is a real-valued scalar field
defined over the spatial extent of a living biological system. It is the
order parameter for consciousness-relevant coherence: its value at a
point $\mathbf{r}$ and time $t$ encodes the degree to which the local
biological state participates in a globally organized, dynamically coherent
pattern.

Several features of this definition require emphasis.

\paragraph{Real-valued and scalar.}
The Loomfield is a single real number at each point in space and time, not
a vector or tensor. This is a simplification: the underlying biological
processes are rich and multi-component. But an effective description
succeeds precisely by discarding irrelevant detail. The scalar character
of $L$ captures the essential feature---the presence or absence, and the
magnitude, of coherent organization---without attempting to encode the full
state of every biological subsystem. This is analogous to the use of a
scalar order parameter in the Ginzburg-Landau theory of
superconductivity, where the complex pair condensate amplitude $\psi$
captures the macroscopic quantum coherence of the superconducting state
without resolving the wavefunction of each Cooper pair.

\paragraph{Spatially extended.}
$L$ is defined everywhere within the organism's volume, not at discrete
points or in a network graph. This reflects the thesis that consciousness
is a field-level phenomenon: it has a spatial structure, with regions of
high and low coherence, wavefronts, gradients, and interference patterns.
The spatial extent of coherent $L$---the volume over which $L$ maintains
an organized, non-random pattern---corresponds, in CLT, to the domain of
integrated conscious experience.

\paragraph{Temporally dynamic.}
$L$ evolves in time. Conscious experience is not static; it is a
continuously evolving process with a characteristic temporal grain.
The dynamics of $L$---its oscillations, propagation, and decay---are
what distinguish a conscious biological system from one that is merely
alive. A system in deep coma may be biologically active, but if
$L(\mathbf{r}, t)$ lacks coherent temporal dynamics (i.e.,
$\partial L / \partial t \to 0$ despite nonzero $L$), the system is not
conscious in the sense CLT formalizes.

\paragraph{Emergent, not fundamental.}
The Loomfield does not correspond to a new physical force, a new quantum
field, or an undiscovered form of radiation. It is a coarse-grained
description of collective biological dynamics, in the same sense that
temperature is a coarse-grained description of molecular kinetic energy.
Temperature is real---it has causal consequences, it can be measured, it
obeys its own laws (thermodynamics)---but it does not appear in the
fundamental Lagrangian of particle physics. Similarly, the Loomfield is
real in the sense that it captures genuine, measurable structure in
biological systems, and it obeys its own dynamical equations
(Section~\ref{sec:viable-window}), but it is not a new entry in the
Standard Model.

\subsection{What the Loomfield Encodes}

The Loomfield, as an effective field, aggregates information from the
biological substrates into a single description of coherence-relevant
macroscopic state. Specifically, coherent excitations of $L$ encode four
properties of the underlying biology:

\begin{enumerate}
    \item \textbf{Spatial integration.}  A coherent Loomfield pattern---one
    with high spatial autocorrelation and low roughness---indicates that
    biological processes across distant tissue regions are coordinated.
    This is the physical counterpart of the experiential observation that
    consciousness is unified: a coherent $L$ means that what happens at
    one point in the organism is constrained by, and informationally
    connected to, what happens at other points.

    \item \textbf{Temporal persistence.}  A sustained Loomfield excitation
    indicates that coherent dynamics persist despite the relentless
    molecular turnover characteristic of living systems. Biological matter
    is replaced on timescales of days to weeks, yet experiential continuity
    is maintained. The Loomfield captures this persistence at the level of
    organized pattern rather than material identity.

    \item \textbf{Decoherence resistance.}  A nonzero Loomfield amplitude
    in the presence of thermal noise, metabolic fluctuation, and
    environmental perturbation indicates that the system is actively
    maintaining coherence against dissipative forces. This active
    maintenance---requiring continuous energetic input---is what
    distinguishes a conscious living system from a dead one with the same
    chemical composition.

    \item \textbf{Dynamic complexity.}  Conscious experience is neither
    static nor random; it is structured and evolving. The temporal
    derivative $\partial L / \partial t$ and the spatial gradient
    $\nabla L$ together characterize the richness of the field's dynamics.
    A field that is coherent but frozen ($\partial L / \partial t = 0$)
    lacks the informational flow associated with consciousness; a field
    that is rapidly fluctuating but spatially disorganized lacks the
    integration. Both coherence \emph{and} dynamics are required.
\end{enumerate}

\subsection{Distinction from Other Field-Based Proposals}

CLT's Loomfield differs from previous field-based theories of consciousness
in several respects that warrant explicit clarification.

\paragraph{Distinction from CEMI.}
\citet{mcfadden2020cemi} proposes that the brain's endogenous
electromagnetic (EM) field is the physical substrate of conscious
experience. The CEMI field is a specific physical field---it is the EM
field as described by Maxwell's equations, generated by neuronal ion
currents and measurable by EEG or MEG. The Loomfield is not the EM
field, nor any other single physical field. It is an \emph{effective}
description that aggregates contributions from multiple physical
substrates---bioelectric fields (which do involve EM effects), biophoton
emission (optical), microtubule oscillations (mechanical and possibly
quantum), and genetic constraints (biochemical). The EM field is one
contributor to the Loomfield's source term, not the Loomfield itself.
This distinction matters because it allows CLT to incorporate substrate
contributions that have no electromagnetic character---microtubule
lattice phonons, for instance, or epigenetic modulation of coherence
capacity---within a single framework.

\paragraph{Distinction from computational and informational approaches.}
IIT \citep{tononi2016iit} and GNW \citep{dehaene2011gnw} describe
consciousness in terms of information integration or global access,
respectively. These are abstract, substrate-neutral descriptions: they
apply equally to silicon and to neurons, in principle. The Loomfield in CLT v1.1, by
contrast, is a \emph{physical} field defined over biological matter. It
has a propagation velocity, a damping rate, and spatial structure that
depend on the material properties of the tissue in which it is defined.
This is a feature, not a limitation: CLT is designed to make contact with
experimental biology, and a physical field description is what allows the
theory's quantities to be measured rather than merely computed.

\paragraph{Distinction from quantum consciousness theories.}
A family of proposals---most prominently Orch~OR
\citep{hameroff2014consciousness}, but also quantum brain dynamics
(Vitiello, Jibu \& Yasue) and quantum cognition models---identify quantum
coherence, entanglement, or collapse as the essential mechanism producing
consciousness. These theories share a common structure: they locate the
origin of conscious experience in a specific quantum process occurring in a
specific biological substrate, and they typically require that quantum
coherence be maintained at biologically relevant scales despite thermal
decoherence.

CLT takes a fundamentally different stance on this question. The Loomfield
is a \emph{classical effective field}. Its wave equation
(Eq.~\ref{eq:loomfield-wave}) is a classical PDE; its coherence metrics are
statistical measures of spatial and temporal correlation; its dynamics are
those of a driven, dissipative system, not of a quantum superposition. This
does not mean CLT denies quantum effects in biology. Microtubule
oscillations---one of the four Loomfield source substrates
(Section~\ref{sec:substrates})---may well involve quantum coherence at the
molecular scale, and if they do, those quantum effects would contribute to
$\rho_{\mathrm{coh}}$ through the microtubule coupling term. But the
Loomfield itself is defined at a coarse-grained scale where quantum
decoherence has already occurred and the relevant order is
\emph{classical collective coherence}: spatial correlations, phase
synchronization, and organized pattern dynamics. The relationship is
analogous to that between quantum mechanics and classical thermodynamics:
the latter emerges from the former, but its laws are stated and tested at
the macroscopic level without requiring resolution of quantum states.

This architecture makes CLT robust to the outcome of ongoing debates about
quantum biology. If Orch~OR's gravitational objective reduction is confirmed
experimentally, it would enrich the microtubule source term; if it is
falsified, the remaining substrates still contribute and the Loomfield
framework remains intact. If quantum coherence in microtubules is found to
decohere too rapidly to be biologically relevant, the classical oscillatory
dynamics of the tubulin lattice still provide a meaningful source
contribution. CLT's predictions do not depend on any single substrate
mechanism being quantum-mechanical in nature. This modularity is a
deliberate design choice: it insulates the framework's testable predictions
from unresolved foundational questions in quantum biology.

\subsection{The Loomfield and Its Governing Dynamics}

Having defined the Loomfield conceptually, we can state the form of
its dynamics. The Loomfield obeys a sourced, damped wave equation:
%
\begin{equation}
    \nabla^2 L \;-\; \frac{1}{v_L^2}\,\frac{\partial^2 L}{\partial t^2}
    \;-\; \frac{\gamma}{v_L^2}\,\frac{\partial L}{\partial t}
    \;=\; \kappa_L \cdot \rho_{\mathrm{coh}}(\mathbf{r}, t)\,.
    \label{eq:loomfield-wave}
\end{equation}
%
Each term has a specific physical interpretation. The wave operator
$\nabla^2 L - v_L^{-2}\,\partial^2_t L$ ensures that coherent spatial
patterns---standing waves, propagating fronts, resonant modes---are natural
solutions, consistent with the spatially extended character of conscious
experience and with the observation that neural and somatic activity
involves propagating oscillatory patterns across cortical and subcortical
structures \citep{buzsaki2004neuronal}. The damping term
$-(\gamma / v_L^2)\,\partial L / \partial t$ ensures that coherence is
not self-sustaining: without active driving, the field decays. This is
the formal expression of the empirical fact that consciousness requires
metabolic support. The source term
$\kappa_L \cdot \rho_{\mathrm{coh}}$ couples the field to its biological
substrates through the coherence source density $\rho_{\mathrm{coh}}$,
which aggregates contributions from bioelectric, biophotonic, microtubule,
and genetically constrained processes (Section~\ref{sec:substrates}).

The parameters $v_L$ (effective propagation velocity), $\gamma$ (damping
coefficient), and $\kappa_L$ (coupling constant) are not arbitrary. They
are constrained by the requirement that Loomfield dynamics reproduce
observable features of biological coherence: the spatial scale of
cortical coherence patterns, the temporal scale of conscious transitions,
and the metabolic cost of maintaining wakefulness. Section~6 discusses
how these parameters connect to measurable quantities.

The wave equation structure is a \emph{choice}, not a derivation from first
principles. We adopt it because wave equations are the simplest partial
differential equations that support spatially extended, temporally evolving,
interference-capable patterns---precisely the properties required of a field
description of integrated conscious experience. The damped-and-driven
variant is standard in non-equilibrium physics
\citep{haken1975cooperative}: it describes systems maintained in organized
states by a balance between dissipation and driving, which is the physical
situation of a living organism. Should future theoretical work derive the
Loomfield dynamics from a more fundamental principle (e.g., a variational
principle or a renormalization group flow from substrate-level descriptions),
the wave equation form would appear as the leading-order effective
description---just as the Navier-Stokes equations emerge as the effective
dynamics of molecular fluids.

\subsection{Summary and Transition}

The Loomfield is the central theoretical construct of CLT. At the biological scale, it is an
effective scalar field that describes the coherence-relevant macroscopic
dynamics of a biological system, encoding spatial integration, temporal
persistence, decoherence resistance, and dynamic complexity. It is not a
new fundamental field, but an emergent collective variable in the tradition
of condensed-matter order parameters. It obeys a damped wave equation
sourced by the coherent activity of biological substrates, ensuring that
consciousness requires both spatial organization and active energetic
maintenance.

With the Loomfield defined, the next question is: under what conditions
do its dynamics support the coherent excitations that CLT identifies with
consciousness? Section~3 addresses this question by introducing the Energy
Resistance parameter and the viable energetic window---the region of
parameter space within which coherent Loomfield dynamics are sustainable.

\section{Coherence Regimes and the Viable Energetic Window}
\label{sec:viable-window}

\subsection{Coherence as an Actively Maintained Regime}

The Loomfield wave equation (Eq.~\ref{eq:loomfield-wave}) contains a
damping term. This is not a mathematical convenience; it is the central
physical feature of the theory. The damping term $-(\gamma / v_L^2)\,
\partial L / \partial t$ means that every coherent excitation of the
Loomfield is transient: in the absence of active sourcing
($\rho_{\mathrm{coh}} = 0$), the field decays exponentially toward zero.
Consciousness, in CLT, is not a property that a system \emph{has}; it is a
regime that a system \emph{maintains}, continuously, against dissipation.

This perspective has a precise physical lineage. \citet{schrodinger1944life}
observed that living systems maintain themselves in organized states by
``feeding on negative entropy''---importing free energy from the
environment to offset the thermodynamic tendency toward disorder.
\citet{prigogine1977self} formalized this insight: organized structures in
systems driven far from thermodynamic equilibrium---convection cells,
chemical oscillations, biological morphogenesis---exist only while the
driving persists. They are \emph{dissipative structures}: patterns that
arise from, and are sustained by, a continuous flux of energy through the
system. Shut off the flux, and the structure vanishes.

The Loomfield is a dissipative structure in this sense. Its coherent
excitations---the spatial patterns, temporal oscillations, and propagating
wavefronts that CLT identifies with consciousness---exist only while the
biological substrates actively source them. When metabolic supply is
interrupted (ischemia), when substrate dynamics are pharmacologically
suppressed (anesthesia), or when the organism dies, the source density
$\rho_{\mathrm{coh}}$ diminishes and the Loomfield decays. The damping
rate $\gamma$ quantifies how quickly: larger $\gamma$ means faster decay,
meaning greater metabolic cost to maintain a given level of coherent
excitation.

But dissipative maintenance alone is not sufficient. Not every driven
system is coherent; not every energized organism is conscious. The critical
question is: under what conditions does the balance between driving and
dissipation produce \emph{organized}, spatially integrated, dynamically
rich Loomfield excitations rather than mere noise or rigid stasis? CLT
answers this question through the concept of the viable energetic window.

\subsection{Energy Resistance and the Phase Space of Viability}

The relationship between energetic throughput and dynamical organization is
not monotonic. Too little energy, and the system cannot maintain coherence
against thermal and environmental noise---it decoheres into disorder. Too
much energy relative to the system's capacity for organization, and the
dynamics become rigid or locked into fixed patterns that cannot adapt. This
bounded relationship is formalized through the \emph{Energy Resistance}
parameter, adapted from \citeauthor{picard2025erp}'s Energetic Resistance
Principle \citep{picard2025erp}:
%
\begin{equation}
    \acute{e}R \;=\; \frac{\mathrm{EP}}{f^2}\,,
    \label{eq:energy-resistance}
\end{equation}
%
where $\mathrm{EP}$ denotes the \emph{energy present}---the available
energetic capacity of the system, incorporating metabolic rate, pattern
energy (the energetic cost of maintaining non-equilibrium states), and
substrate coupling strength---and $f$ denotes the dominant \emph{frequency}
of system dynamics, reflecting the rate of energetic throughput and
oscillatory activity.

Equation~\ref{eq:energy-resistance} defines a two-dimensional phase space
$(\mathrm{EP}, f)$ in which constant-$\acute{e}R$ curves are parabolas
$\mathrm{EP} = \acute{e}R \cdot f^2$. Each point in this phase space
represents a particular energetic configuration of the biological system.
The viable window is the region between two critical thresholds:
%
\begin{equation}
    \acute{e}R_{\min} \;\leq\; \acute{e}R \;\leq\; \acute{e}R_{\max}\,,
    \label{eq:viable-bounds}
\end{equation}
%
with the boundaries delineating three qualitatively distinct regimes.

\subsection{Three Regimes}

\paragraph{The chaos regime ($\acute{e}R < \acute{e}R_{\min}$).}
When the ratio of energy to frequency-squared falls below the lower
threshold, the system's dynamical rate overwhelms its capacity for
structured organization. Energy throughput is high relative to the system's
organizational bandwidth: oscillatory activity is too fast, too strong, or
too poorly constrained for the substrates to maintain coordinated patterns.
The Loomfield source density $\rho_{\mathrm{coh}}$ becomes spatially
fragmented and temporally erratic. The spatial coherence metric $Q$
collapses toward zero, and the consciousness observable $C_{\mathrm{bio}}$
vanishes despite substantial total field energy.

The biological correlates of this regime are states characterized by
excessive, disorganized excitation: epileptic seizure, excitotoxic injury,
high fever, psychotic mania. In each case, the system is not
under-energized; it is \emph{over-driven}---energy throughput exceeds the
organizational capacity of the substrates, producing activity without
integration.

\paragraph{The viable window
($\acute{e}R_{\min} \leq \acute{e}R \leq \acute{e}R_{\max}$).}
Within the bounded region, the balance between energetic driving and
dissipation supports coherent Loomfield excitations. The substrates
generate $\rho_{\mathrm{coh}}$ with sufficient spatial organization and
temporal stability to source propagating, interfering, and resonating
field patterns. The Loomfield exhibits the four properties identified in
Section~\ref{sec:loomfield}: spatial integration, temporal persistence,
decoherence resistance, and dynamic complexity.

This is not a single point but a \emph{region}---the viable window has
interior structure. Different conscious states occupy different locations
within it. Resting wakefulness, focused attention, REM sleep, deep
meditation, creative flow, and physical exercise all correspond to
distinct $(\mathrm{EP}, f)$ coordinates within the window. They differ in
their energetic configuration and dynamical character, but they share the
property that their Loomfield dynamics are coherent and actively
maintained. The window's geometry is asymmetric: it narrows at high
frequencies (fast dynamics are harder to organize) and broadens at low
frequencies (slow, low-energy states have a wider margin of viability).
This asymmetry has consequences for pathology, as discussed below.

\paragraph{The rigidity regime ($\acute{e}R > \acute{e}R_{\max}$).}
When $\acute{e}R$ exceeds the upper threshold, the system is
over-constrained relative to its dynamical rate. Energy is present, but
the dynamics are too slow, too suppressed, or too stereotyped to support
the adaptive, evolving patterns characteristic of consciousness. The
Loomfield may maintain nonzero amplitude, but its temporal derivatives
approach zero: $\partial L / \partial t \to 0$. The field becomes
static---organized, perhaps, but frozen. The consciousness observable
$C_{\mathrm{bio}}$, which requires both coherence \emph{and} active
dynamics, vanishes even though $Q$ may remain moderate.

The biological correlates are states of excessive constraint or suppression:
severe depression (reduced dynamical range, affective flattening),
catatonia, deep hypothermia, advanced neurodegeneration. In these
conditions, the system retains structural integrity but lacks the dynamic
flexibility that consciousness requires. The system is not disorganized; it
is \emph{locked}.

\subsection{Phase-Space Geometry and Regime Boundaries}

The three regimes are not discrete categories separated by sharp walls.
The boundaries $\acute{e}R_{\min}$ and $\acute{e}R_{\max}$ are thresholds
in a continuous phase space, and biological systems can approach, cross,
and recede from them gradually. This continuity has important consequences.

First, \emph{borderline states} exist. A system near $\acute{e}R_{\min}$
is vulnerable to chaotic decompensation under perturbation (a small increase
in excitatory drive could push it across the boundary), while a system near
$\acute{e}R_{\max}$ is vulnerable to rigidity under further constraint.
Anxiety, for instance, may be understood as a state near the chaos boundary:
the system remains within the viable window but with diminished margin,
producing subjective fragility and hypervigilance. Conversely, mild
depression may correspond to proximity to the rigidity boundary: still
conscious, but with reduced dynamical range and a characteristic
experiential flatness.

Second, \emph{trajectories through phase space} are meaningful. A biological
system does not occupy a fixed point in $(\mathrm{EP}, f)$ space; it traces
a trajectory as metabolic state, arousal level, circadian phase, and
environmental conditions change. The daily cycle of waking, sleep, and
dreaming corresponds to a closed orbit within (and occasionally touching
the boundaries of) the viable window. Recovery from illness, the onset of
anesthesia, and the progression of neurodegenerative disease each trace
characteristic paths. A seizure is a rapid excursion across
$\acute{e}R_{\min}$ into the chaos regime, followed (in a self-limiting
seizure) by a return. Death is a terminal trajectory toward
$\rho_{\mathrm{coh}} = 0$, in which the system exits the viable window
and the Loomfield decays irreversibly.

Third, the \emph{viable window itself is not fixed}. Its boundaries depend
on the substrate properties of the individual organism: the density and
connectivity of gap junctions, the health of mitochondrial function, the
integrity of microtubule lattices, and the genetic and epigenetic
constraints on coherence capacity. Interventions that improve substrate
function---restoring mitochondrial health, enhancing gap junction
coupling, reducing chronic neuroinflammation---can \emph{widen} the viable
window, increasing the organism's margin of viability. Conversely, aging,
disease, and chronic stress can \emph{narrow} it, making the system more
vulnerable to boundary crossing. This is not a metaphor; it is a
quantitative claim about the dependence of $\acute{e}R_{\min}$ and
$\acute{e}R_{\max}$ on substrate parameters, which
Section~\ref{sec:substrates} develops in detail.

\subsection{Health and Pathology as Phase-Space Position}

The viable-window framework reframes the relationship between health and
pathology. In CLT, a ``healthy'' conscious state is not defined by the
absence of disease but by the system's position well within the interior
of the viable window, with sufficient margin from both boundaries.
Pathology is not a binary: it is proximity to, or transgression of, a
regime boundary. This has several implications.

Different pathologies correspond to \emph{different directions} of
departure from the viable window. Seizure and mania are exits toward the
chaos boundary; depression and catatonia are exits toward the rigidity
boundary. These are physically distinct processes---they involve different
substrate failures and require different interventions---even though both
result in diminished consciousness.

Recovery is a trajectory, not a switch. A system that has crossed into the
chaos regime does not return instantaneously to the viable interior; it
must traverse the boundary region, and the path it takes may involve
transient passages through states that are themselves suboptimal. The
clinical observation that recovery from psychosis or severe depression is
gradual, non-monotonic, and often involves intermediate symptomatic
states is consistent with a phase-space trajectory that spirals back
toward the viable interior rather than jumping directly to it.

The framework also accommodates \emph{resilience}. Two organisms may
occupy the same $(\mathrm{EP}, f)$ coordinate but differ in their
distance from the nearest boundary---one has a wide viable window (robust
substrates, strong coherence capacity) and the other has a narrow one
(compromised substrates, reduced margin). The first is resilient to
perturbation; the second is vulnerable. This provides a physical
interpretation of the clinical concept of allostatic load: cumulative
stress narrows the viable window, reducing the system's capacity to absorb
further perturbation without crossing a regime boundary.

\subsection{Summary and Transition}

Consciousness, in CLT, is not a static property but an actively maintained
dynamical regime that exists within a bounded region of energetic phase
space. The viable window is defined by the balance between energetic
throughput and organizational capacity, formalized through the Energy
Resistance parameter $\acute{e}R = \mathrm{EP}/f^2$. Below the window,
the system is chaotically over-driven; above it, rigidly under-dynamic.
Within it, the Loomfield sustains the coherent, spatially integrated,
temporally evolving excitations that CLT identifies with consciousness.
Health is interior position; pathology is boundary proximity or
transgression; recovery is a return trajectory.

The viable window's boundaries are not universal constants---they depend on
the properties of the biological substrates that source the Loomfield. We
turn now to those substrates: the bioelectric, biophotonic, microtubule,
and genetic systems whose coordinated activity generates the coherence
source density $\rho_{\mathrm{coh}}$ and determines the region of phase
space in which consciousness is possible.

\section{Biological Substrates Supporting Loomfield Coherence}
\label{sec:substrates}

The Loomfield is an effective field---it describes collective coherence
without specifying which microscopic processes produce it. But an effective
description is only useful if one can identify the degrees of freedom it
aggregates. In condensed-matter physics, the magnetization field aggregates
electron spins; in hydrodynamics, the velocity field aggregates molecular
momenta. The Loomfield aggregates the coherence-relevant dynamics of
biological substrates.

This section identifies four substrate systems that contribute to the
coherence source density $\rho_{\mathrm{coh}}(\mathbf{r}, t)$ appearing in
the Loomfield wave equation (Eq.~\ref{eq:loomfield-wave}). Each substrate
operates at characteristic spatial and temporal scales, and each supports a
distinct aspect of the coherent dynamics that CLT associates with
consciousness. No single substrate generates consciousness; together, they
enable the sustained, spatially integrated, dynamically rich Loomfield
excitations described in Sections~\ref{sec:loomfield}
and~\ref{sec:viable-window}.

\subsection{Bioelectric Field Organization}

Living tissues maintain endogenous voltage gradients---spatially patterned
distributions of resting membrane potential, ion fluxes, and electric fields
that extend across cell populations. These bioelectric patterns are not
merely consequences of neural activity; they are large-scale organizational
signals that coordinate tissue behavior across spatial scales far exceeding
individual cells \citep{levin2014bioelectricity}.

\citet{levin2021bioelectric} has demonstrated that bioelectric signaling
constitutes a reprogrammable control layer in embryogenesis, regeneration,
and oncogenesis. Voltage gradients established by ion channels and
maintained by gap junction networks carry positional and state information
that guides collective cell behavior. Crucially, these patterns exhibit
properties directly relevant to Loomfield coherence: they are spatially
extended (spanning tissue and organ scales), temporally persistent (stable
over periods much longer than single action potentials), and functionally
integrated (changes in one region propagate to and constrain distant
regions through gap junction coupling).

Gap junctions deserve particular emphasis. These intercellular channels
electrically couple adjacent cells, creating syncytial networks through
which voltage signals propagate without synaptic delay. In neural tissue,
gap junction coupling produces synchronized oscillations across neuronal
populations; in non-neural tissue, it maintains the bioelectric gradients
that guide morphogenesis and wound healing
\citep{fields2022morphogenesis}. The density, conductance, and topology of
gap junction networks determine the spatial scale over which bioelectric
coherence can be maintained---and thus, in CLT's formalism, the spatial
extent of the bioelectric contribution to $\rho_{\mathrm{coh}}$.

From CLT's perspective, the bioelectric substrate provides the
\emph{large-scale spatial integration} component of Loomfield coherence.
Bioelectric patterns are the natural candidate for the mechanism by which
distant tissue regions become informationally coupled, producing the
spatially extended coherent $L$ that CLT associates with unified conscious
experience. The characteristic timescales span milliseconds (neural action
potentials and fast oscillations) to hours (developmental bioelectric
gradients), providing a multi-timescale scaffold for coherence maintenance.

\subsection{Biophoton Emission and Optical Signaling}

All living cells emit ultraweak photon radiation---photons in the visible
and near-UV range, at intensities of $10^1$--$10^3$ photons per second per
square centimeter of tissue surface. This phenomenon, variously termed
ultraweak photon emission (UPE), biophoton emission, or biological
autoluminescence, has been measured reproducibly across organisms from
bacteria to humans \citep{vanwijk2005biophoton}. The emission is not
thermal radiation; it originates primarily from electronically excited
species produced during oxidative metabolic processes, particularly those
involving reactive oxygen species and lipid peroxidation
\citep{cifra2011biophoton}.

Several properties of biophoton emission are relevant to CLT. First, the
emission intensity correlates with metabolic activity and redox state,
providing a real-time optical indicator of the energetic processes that
sustain biological organization. Second, the photon statistics of
biophoton emission have been reported to deviate from Poissonian
(random) distributions, with some studies observing super-Poissonian or
coherent-like statistics \citep{popp2003biophotons}, though this remains
an area of active investigation and debate. Third, neural tissue is a
significant source of biophoton emission, and recent work has reported
correlations between biophoton emission and neural activity patterns
\citep{murugan2024biophotons}.

CLT does not claim that biophotons \emph{are} consciousness, nor that
optical signaling is the primary mechanism of neural integration. The claim
is more modest: biophoton emission constitutes a measurable substrate whose
spatial and temporal patterns carry information about the coherence state of
the underlying metabolic and neural processes. To the extent that biophoton
emission patterns reflect coordinated metabolic activity across tissue
regions, they contribute to $\rho_{\mathrm{coh}}$ as an indicator of---and
possibly a participant in---the coherent dynamics that the Loomfield
describes. The biophotonic substrate provides a \emph{rapid, potentially
long-range} coherence channel: photons propagate at the speed of light
through tissue (with scattering and absorption), and their emission
patterns can in principle coordinate or reflect coordination across
spatial scales that ion-based signaling reaches only through multi-synaptic
or gap-junction pathways.

\subsection{Cytoskeletal Dynamics and Intracellular Coherence}

Within individual cells, microtubules---hollow protein polymers composed of
tubulin dimers arranged in a helical lattice---form the primary structural
and organizational scaffold of the cytoskeleton. Beyond their well-known
roles in cell division, intracellular transport, and morphological
maintenance, microtubules exhibit dynamical properties that are relevant to
coherence at the subcellular scale.

\citet{bandyopadhyay2013microtubule} demonstrated that single brain
microtubules support multi-level oscillatory behavior: resonant responses at
multiple frequencies, from kilohertz to megahertz, arising from the
lattice geometry of the tubulin polymer. These oscillations are not
isolated curiosities; they indicate that microtubules function as
frequency-selective resonators whose dynamical behavior depends on
structural parameters (lattice spacing, polymer length, tubulin
conformational state) that are under biological control. Biophysical
modeling confirms that the anisotropic elastic and dielectric properties of
the microtubule lattice support propagating vibrational modes across the
polymer length \citep{tuszynski2004microtubule}.

The Orch~OR proposal \citep{hameroff2014consciousness} identifies quantum
coherence within microtubules as the mechanism of consciousness. CLT takes
a different and deliberately more conservative position. The Loomfield
framework requires only that microtubules contribute \emph{organized
oscillatory dynamics} at the intracellular scale---a claim supported by the
experimental evidence cited above, regardless of whether the underlying
physics is quantum or classical. If quantum coherence contributes to
microtubule dynamics, it would enrich the oscillatory structure of the
microtubule source term; if it does not, the classical vibrational modes
documented by \citeauthor{bandyopadhyay2013microtubule} and others are
sufficient.

From CLT's perspective, the cytoskeletal substrate provides
\emph{intracellular coherence}: the organized oscillatory behavior within
individual cells that seeds larger-scale patterns. Microtubule dynamics
integrate molecular-level processes (protein conformational changes, motor
protein activity, ion interactions) into coherent cellular-level
oscillations. These oscillations, in turn, couple to the bioelectric and
biophotonic substrates through electromagnetic and mechanical interactions,
contributing a fine-grained, high-frequency component to the coherence
source density.

\subsection{Genetic and Epigenetic Constraint Systems}

The three substrates described above---bioelectric, biophotonic, and
cytoskeletal---generate coherence dynamically, on timescales from
milliseconds to hours. The fourth substrate operates on a fundamentally
different timescale: it does not generate coherence but \emph{constrains}
the space of possible coherence regimes.

The genome specifies the proteins---ion channels, gap junction connexins,
tubulin isoforms, metabolic enzymes---from which the coherence-generating
substrates are constructed. Different organisms, and different cell types
within an organism, express different repertoires of these molecular
components, producing different substrate properties and therefore different
coherence capacities. The genetic substrate does not source
$\rho_{\mathrm{coh}}$ directly; rather, it determines the parameter ranges
within which $\rho_{\mathrm{coh}}$ can operate. In the language of
Section~\ref{sec:viable-window}, the genome shapes the boundaries of the
viable window.

Epigenetic regulation adds a layer of adaptive modulation.
\citet{waddington1957epigenetic} introduced the concept of the epigenetic
landscape to describe how identical genomes give rise to different
stable cell fates through differential gene regulation. Modern molecular
biology has elaborated this concept into a detailed understanding of
chromatin modification, DNA methylation, and non-coding RNA regulation
\citep{allis2016epigenetics}. These mechanisms modulate gene expression on
timescales of hours to years, altering the availability of coherence-relevant
proteins without changing the underlying DNA sequence.

For CLT, epigenetic modulation provides a mechanism by which the viable
window can shift in response to experience, environment, and developmental
stage. Chronic stress, for example, produces epigenetic changes that alter
ion channel expression and gap junction density---modifications that, in
CLT's framework, correspond to a narrowing of the viable window (reduced
$\acute{e}R_{\max} - \acute{e}R_{\min}$). Conversely, enriched
environments and sustained healthy metabolic conditions may produce
epigenetic states that widen it. Developmental transitions---from embryo
to neonate to adult to senescence---involve large-scale epigenetic
reprogramming that reconfigures the substrate landscape and, with it, the
coherence capacities available to the Loomfield.

This is a constraint role, not a generative one, but it is essential to
the theory. Without genetic and epigenetic specification of substrate
properties, the three dynamical substrates would have no defined operating
parameters, and the viable window would have no boundaries. The genetic
substrate provides the \emph{constitutional constraints} within which
coherence dynamics unfold.

\subsection{Substrate Integration and the Coherence Source Density}

The four substrates do not operate in isolation. They interact across
spatial and temporal scales in a manner that CLT formalizes through the
coherence source density:
%
\begin{equation}
    \rho_{\mathrm{coh}}(\mathbf{r}, t) \;=\;
    \sum_{i} w_i \, S_i(\mathbf{r}, t)\,,
    \label{eq:rho-coh}
\end{equation}
%
where $S_i$ denotes the coherence contribution of the $i$-th substrate
(bioelectric, biophotonic, cytoskeletal, genetic/epigenetic) and $w_i$ is
a coupling weight reflecting that substrate's relative contribution at a
given tissue location. The weights $w_i$ are not universal constants; they
vary with tissue type (neural tissue has higher bioelectric and cytoskeletal
weights; metabolically active tissue has higher biophotonic weights), with
developmental stage (the genetic/epigenetic constraint term shifts across
the lifespan), and with physiological state.

The critical point is that $\rho_{\mathrm{coh}}$ is a \emph{superposition}
of substrate contributions. Consciousness, in CLT, does not require any
single substrate to achieve extraordinary performance. It requires the
integrated output of multiple substrates, each contributing its
characteristic spatial and temporal coherence, to collectively source a
Loomfield excitation that is spatially integrated, temporally persistent,
and dynamically rich. This superposition architecture has a practical
consequence: degradation of one substrate can be partially compensated by
the others, providing a degree of robustness and graceful degradation that
a single-substrate theory would not predict.

Table~\ref{tab:substrates} summarizes the four substrates and their
contributions.

\begin{table}[ht]
\centering
\caption{Biological substrates supporting Loomfield coherence.}
\label{tab:substrates}
\begin{tabular}{@{}lllp{4.5cm}@{}}
\toprule
\textbf{Substrate} & \textbf{Scale} & \textbf{Timescale} & \textbf{Primary contribution} \\
\midrule
Bioelectric & Tissue--organ & ms--hours & Spatial integration, non-local coupling \\
Biophotonic & Cell--tissue & ns--seconds & Rapid signaling, metabolic coherence indicator \\
Cytoskeletal & Molecular--cell & $\mu$s--ms & Intracellular coherence, frequency-selective resonance \\
Genetic/epigenetic & Genome-wide & Hours--lifespan & Constitutional constraints on viable window boundaries \\
\bottomrule
\end{tabular}
\end{table}

\subsection{Summary and Transition}

The Loomfield is sourced by the coordinated activity of four biological
substrate systems. Bioelectric fields provide large-scale spatial
integration through gap junction networks and voltage gradients.
Biophoton emission contributes a rapid, metabolically coupled coherence
channel. Cytoskeletal dynamics generate organized intracellular oscillations
that seed larger-scale patterns. Genetic and epigenetic systems constrain
the parameter space within which the dynamical substrates operate,
determining the boundaries of the viable window.

No single substrate is sufficient; their integration, formalized through the
coherence source density $\rho_{\mathrm{coh}}$, is what enables the
sustained, spatially extended, dynamically complex Loomfield excitations
that CLT identifies with consciousness. The multi-substrate architecture
provides robustness through redundancy and predicts specific patterns of
graceful degradation when individual substrates are impaired.

With the substrates identified, the next question concerns the temporal
structure of their integration. How do processes operating on timescales
from microseconds to years coordinate to maintain a continuously coherent
Loomfield? Section~\ref{sec:temporal} addresses this through the framework
of non-equilibrium stability and multi-timescale dynamics.

\section{Temporal Structure and Non-Equilibrium Stability}
\label{sec:temporal}

The preceding sections have established what the Loomfield is
(Section~\ref{sec:loomfield}), under what energetic conditions it can be
sustained (Section~\ref{sec:viable-window}), and which biological substrates
source it (Section~\ref{sec:substrates}). But these descriptions are
primarily spatial and structural. Consciousness is not a snapshot; it is a
\emph{process}---extended in time, continuously evolving, and dependent at
every moment on the active maintenance of coherent dynamics against
dissipation. This section addresses the temporal dimension of Loomfield
coherence: how it persists, how it is stabilized, and how it recovers after
perturbation.

\subsection{Time as a Defining Dimension of Biological Coherence}

A crystal is spatially ordered but temporally static. A gas is temporally
active but spatially disordered. A living, conscious biological system is
neither: it maintains spatial order \emph{that evolves in time}. The
Loomfield must be coherent in both space and time simultaneously---this is
what the consciousness observable $C_{\mathrm{bio}}$ captures through its
dependence on both the spatial coherence metric $Q$ and the temporal
derivative $\partial L / \partial t$.

Temporal coherence, in this context, does not mean temporal constancy. A
system whose Loomfield is constant in time ($\partial L / \partial t = 0$)
is not temporally coherent; it is frozen, and $C_{\mathrm{bio}}$ vanishes.
Temporal coherence means that successive states of the Loomfield are
\emph{related}---each moment's configuration bears a structured,
non-random relationship to the moments before and after it. The field
evolves, but it evolves along trajectories that reflect organized dynamics
rather than noise. This is the temporal analogue of spatial coherence: just
as spatial coherence means that distant points are correlated, temporal
coherence means that successive moments are correlated.

The damped wave equation (Eq.~\ref{eq:loomfield-wave}) naturally produces
this kind of structured temporal evolution. Propagating waves, standing
modes, and resonant oscillations all generate temporal sequences in which
each moment is determined by the preceding dynamics and the ongoing
driving. The damping term ensures that these sequences are transient
without driving, while the source term ensures that they are continuously
renewed. The result is a field that is neither static nor random but
dynamically organized---precisely the temporal character that CLT ascribes
to conscious experience.

\subsection{Non-Equilibrium Stability and Dissipative Organization}

The stability of Loomfield coherence is not the stability of equilibrium.
An equilibrium system is stable because it occupies a free-energy minimum:
perturbations are restored by thermodynamic forces. A living system is far
from thermodynamic equilibrium; its organized state is not a free-energy
minimum but a continuously maintained configuration sustained by energy
throughput \citep{prigogine1977self}. Remove the energy flux, and the
system decays toward equilibrium---which, for a biological organism, is
death.

This distinction is fundamental to CLT. The Loomfield is stable not because
it occupies a potential minimum but because the balance between driving
(the source density $\rho_{\mathrm{coh}}$) and dissipation (the damping
$\gamma$) maintains it in a dynamical steady state. The stability is
\emph{dynamical}: the field persists because it is continuously regenerated,
not because it is resistant to change. This is the defining feature of
dissipative structures \citep{prigogine1977self, haken1975cooperative}---
a convection cell persists not because the fluid resists rearrangement but
because the temperature gradient continuously drives the organized flow.

For the Loomfield, this means that temporal persistence and metabolic cost
are inseparable. Every moment of sustained coherence requires energetic
expenditure. The damping coefficient $\gamma$ quantifies the rate at which
coherent excitations decay in the absence of driving, and therefore the
metabolic cost of maintaining them. The biological reality underlying this
parameter is the observation that the brain consumes approximately 20\% of
the body's metabolic output despite comprising roughly 2\% of its mass---a
disproportionate energetic investment that CLT interprets as the cost of
maintaining coherent Loomfield dynamics against dissipation.

The non-equilibrium character of this stability has a crucial implication:
it permits \emph{flexibility}. An equilibrium-stable system resists all
perturbations equally; a non-equilibrium-stable system can accommodate
perturbations by adjusting its dynamical balance. The Loomfield can shift
its spatial pattern, alter its dominant frequencies, and reconfigure its
mode structure in response to changing conditions---all while remaining
within the viable window. This adaptive flexibility is what distinguishes
the viable-window regime from the rigidity regime
(Section~\ref{sec:viable-window}): both maintain temporal order, but only
the viable regime maintains order that can change.

\subsection{Oscillatory Stabilization and Rhythmic Structure}

Biological systems are pervasively oscillatory. Neural activity is organized
into rhythmic bands (delta, theta, alpha, beta, gamma) spanning frequencies
from below 1~Hz to above 100~Hz \citep{buzsaki2004neuronal}. Circadian
rhythms impose a 24-hour periodicity on metabolic and physiological
processes \citep{reppert2002circadian}. Cardiac, respiratory, and
hormonal cycles add further temporal structure. These oscillations are not
epiphenomena; they are \emph{stabilizing mechanisms} for non-equilibrium
organization.

The physical basis for oscillatory stabilization is well understood in
nonlinear dynamics. Coupled oscillators synchronize spontaneously when
their coupling strength exceeds a threshold relative to their frequency
dispersion---a phenomenon formalized by the Kuramoto model
\citep{kuramoto1984chemical} and extensively documented in biological
systems \citep{strogatz2003sync}. Synchronized oscillations create
temporal correlations that extend across spatial scales: when two distant
neural populations oscillate in phase, they are temporally coherent even
if no single signal propagates between them. Phase synchronization is thus
a mechanism for achieving the temporal coherence that the Loomfield
requires.

Recent developments in condensed-matter physics provide a sharper
vocabulary for this kind of persistent temporal order.
\citet{wilczek2012timecrystal} introduced the concept of \emph{time
crystals}: systems that spontaneously break time-translation symmetry,
exhibiting stable periodic dynamics in their lowest-energy or steady
state, analogous to the way spatial crystals break continuous spatial
symmetry. While Wilczek's original equilibrium proposal was subsequently
constrained by no-go theorems, \citet{else2016floquet} demonstrated that
\emph{Floquet time crystals}---periodically driven, non-equilibrium
systems---can exhibit robust discrete time-translation symmetry breaking:
the system responds at a period that is a multiple of the driving period,
and this response is stable against perturbation. The essential
ingredients are continuous driving, dissipation, and nonlinear
interactions---precisely the ingredients present in a metabolically active
biological system. CLT does not claim that living organisms are time
crystals in the strict condensed-matter sense. The claim is that
biological oscillatory dynamics exhibit \emph{time-crystal--like}
behavior: persistent, phase-stable temporal periodicity maintained by
continuous metabolic driving, robust to thermal noise and molecular
turnover. Mitochondrial ATP production, redox cycling, and ion pump
activity constitute a biological analogue of Floquet driving---a
sustained, approximately periodic energy input that maintains the system
far from equilibrium and enables the emergence of stable temporal order
that would not exist without it.

In CLT's framework, oscillatory dynamics contribute to Loomfield stability
in three ways. First, they provide a \emph{temporal scaffold}: the
rhythmic structure of neural and metabolic oscillations creates a
predictable temporal framework against which coherent excitations can be
maintained and renewed. Second, they enable \emph{cross-scale coupling}:
slow oscillations (circadian, respiratory) modulate the amplitude and
phase of faster oscillations (neural gamma, microtubule resonances),
creating a nested temporal hierarchy that links the substrates described
in Section~\ref{sec:substrates} across their disparate timescales. Third,
they provide \emph{error correction}: oscillatory systems that drift out
of phase tend to resynchronize (within limits), providing a natural
restoring force that maintains temporal coherence against perturbation.

The multi-timescale character of biological oscillation is particularly
important. The four Loomfield substrates operate on timescales spanning
roughly nine orders of magnitude---from microsecond microtubule vibrations
to year-scale epigenetic remodeling (Table~\ref{tab:substrates}). The
coherence source density $\rho_{\mathrm{coh}}$ integrates across these
timescales. Oscillatory coupling provides the mechanism by which fast and
slow processes coordinate: slow rhythms gate the activity windows of fast
ones, while fast rhythms provide the temporal resolution necessary for
dynamic complexity. The result is a temporal structure that is
simultaneously stable (anchored by slow rhythms) and dynamically
rich (articulated by fast ones).

The time-crystal framework suggests a more specific characterization of
this nested temporal architecture. Experimental work on isolated brain
microtubules has revealed self-similar resonant oscillations recurring at
intervals of approximately three orders of magnitude---from hertz through
kilohertz, megahertz, and beyond---with the pattern structure at each
scale resembling that at adjacent scales
\citep{saxena2022polyatomic, bandyopadhyay2013microtubule}. This
\emph{polyatomic time-crystal--like} behavior---multiple distinct
persistent periodicities coexisting in a single driven structure, organized
in a fractal or scale-invariant hierarchy---represents a ``clocks within
clocks'' architecture in which each timescale hosts its own stable
oscillatory mode, and the modes at different scales are coupled but not
rigidly locked. Such an architecture has a natural stabilizing function:
perturbation at one temporal scale need not propagate to all others,
because each level of the hierarchy has its own phase stability. A
disruption of megahertz-scale microtubule resonances, for instance, need
not immediately destabilize the slower bioelectric oscillations that anchor
large-scale spatial coherence, providing a degree of temporal
compartmentalization that enhances robustness.

CLT incorporates this picture without depending on it. Time-crystal--like
dynamics in microtubules or other substrates, if confirmed, would
constitute one particularly elegant mode of temporal stabilization---one
that naturally explains how coherence is maintained across disparate
timescales. But the Loomfield framework requires only that the biological
substrates collectively maintain temporally organized dynamics within the
viable window; the specific mechanism of temporal stabilization
(Kuramoto-type synchronization, Floquet-like time-crystal behavior,
or other nonlinear dynamical processes) is a question for experimental
resolution. The theory is designed to accommodate whichever mechanisms
biology employs, including combinations of several operating
simultaneously at different scales.

\subsection{Temporal Collapse, Recovery, and Re-Coherence}

If Loomfield coherence is a non-equilibrium steady state maintained by
continuous driving, then interruptions of driving produce temporal
collapse: the field decays, temporal correlations are lost, and
$C_{\mathrm{bio}}$ diminishes. The rate of collapse depends on the damping
coefficient $\gamma$ and the nature of the disruption.

Acute interruptions---cardiac arrest, bolus anesthesia induction, generalized
seizure---produce rapid temporal collapse. The Loomfield decays on a
timescale of order $1/\gamma$, which must correspond to the clinically
observed timescale of consciousness loss (seconds to tens of seconds for
ischemia, seconds for fast-acting anesthetic agents). This provides a
constraint on $\gamma$ that connects the theoretical parameter to
observable physiology (Section~\ref{sec:operationalization}).

Recovery from temporal collapse is not instantaneous reversal. When driving
is restored---circulation resumes, anesthetic is cleared, the seizure
terminates---the Loomfield must rebuild coherence from a decayed or
disordered state. This rebuilding is a \emph{re-coherence process}: the
substrates resume their sourcing, oscillatory coupling re-establishes
temporal correlations, and spatial coherence gradually extends across the
tissue volume. The clinical observation that recovery of consciousness
after cardiac arrest or anesthesia is progressive---passing through stages
of reflex activity, disoriented arousal, and gradually improving cognitive
integration---is consistent with a re-coherence process that proceeds from
local to global, from fast timescales to slow ones.

Chronic degradation presents a different temporal pattern. In
neurodegenerative disease, the substrates deteriorate gradually: synaptic
connections thin, mitochondrial function declines, gap junction density
decreases. The Loomfield does not collapse suddenly; instead, the viable
window narrows as substrate capacity diminishes. The system compensates---
remaining substrates increase their relative contribution, oscillatory
coupling adjusts---until the narrowed window can no longer accommodate
the energetic demands of coherent dynamics, and consciousness degrades.
This is not a single event but a trajectory through the $(\mathrm{EP}, f)$
phase space described in Section~\ref{sec:viable-window}, one in which the
viable region itself contracts around the system's operating point.

The concept of allostatic load \citep{mcewen1998allostatic} finds a natural
interpretation in this framework. Cumulative stress, chronic inflammation,
and metabolic dysfunction progressively degrade substrate function, narrowing
the viable window. Each insult alone may be accommodated, but their
accumulation reduces the margin between the system's operating point and
the nearest regime boundary. Resilience, conversely, corresponds to
substrate conditions that maintain a wide viable window, providing temporal
stability against perturbation.

\subsection{Summary and Transition}

Consciousness, in CLT, is a temporally extended process sustained by
non-equilibrium dynamics. The Loomfield persists not because it is
thermodynamically stable but because it is continuously regenerated by
biological substrates at a rate that exceeds dissipative decay. Oscillatory
and rhythmic activity---potentially exhibiting time-crystal--like
persistence under metabolic driving---provides the temporal scaffold that
stabilizes this regeneration, coupling processes across timescales from
microseconds to circadian cycles through nested, fractal-like hierarchies. Temporal collapse occurs when driving fails; recovery is a
progressive re-coherence process. Chronic degradation narrows the viable
window, reducing the temporal margin within which coherent dynamics can be
maintained.

The theoretical framework is now substantially complete: the Loomfield as
effective field (Section~\ref{sec:loomfield}), its viable energetic regime
(Section~\ref{sec:viable-window}), its biological substrates
(Section~\ref{sec:substrates}), and its temporal structure (this section).
The question that follows is whether this framework can be made
\emph{empirically tractable}. Section~\ref{sec:operationalization} turns to
operationalization: how the quantities CLT defines connect to measurable
observables, and what experimental designs could test the theory's claims.

\section{Operationalization and Testability}
\label{sec:operationalization}

A theory of consciousness that cannot, in principle, be tested against
empirical data is not a scientific theory. The preceding sections have
developed CLT's theoretical structure: the Loomfield as effective field,
the viable energetic window, the biological substrates, and the temporal
dynamics that sustain coherence. This section addresses whether the
quantities CLT defines can be connected to measurable observables, and what
kinds of experiments could, in principle, confirm or refute the theory's
claims.

\subsection{What Testability Requires}

A testable theory of consciousness must satisfy two conditions. First, it
must define quantities that are \emph{operationally meaningful}: each
theoretical construct must connect, through a specified chain of reasoning,
to something that can be measured with existing or foreseeable experimental
methods. Second, it must make \emph{differential predictions}: it must
predict outcomes that differ from those predicted by competing theories or
by the null hypothesis that the theoretical constructs are epiphenomenal.

CLT's central quantities---the Loomfield $L(\mathbf{r}, t)$, the coherence
source density $\rho_{\mathrm{coh}}$, the spatial coherence metric $Q$,
the Energy Resistance $\acute{e}R$, and the consciousness observable
$C_{\mathrm{bio}}$---are all defined in terms of spatial and temporal
properties of biological dynamics. None requires access to a subject's
phenomenal experience for measurement; all are, in principle, extractable
from physical observations of biological systems. This is a deliberate
design feature: CLT is formulated so that its quantities can be evaluated
from third-person data.

\subsection{Observable Proxies for Loomfield Coherence}

The Loomfield is an effective field; it is not directly measurable in the
way that an electric field is measurable with a voltmeter. But its
properties---spatial integration, temporal persistence, decoherence
resistance, dynamic complexity---have observable proxies in existing
measurement modalities.

\paragraph{Spatial coherence.}
The spatial coherence metric $Q$ quantifies the degree to which the
Loomfield maintains organized, non-random spatial structure. Observable
proxies include long-range phase synchronization in EEG or MEG signals
\citep{buzsaki2004neuronal}, spatial correlation structure in fMRI BOLD
signals, and the spatial extent of coordinated bioelectric activity
measured by high-density electrode arrays. The prediction is specific:
$Q$ should be high during conscious states and low during states of
diminished consciousness (deep sleep, anesthesia, coma), and transitions
between these states should exhibit continuous $Q$ dynamics rather than
abrupt switches.

\paragraph{Temporal dynamics.}
The temporal derivative $\partial L / \partial t$ enters the consciousness
observable $C_{\mathrm{bio}}$ directly. Observable proxies include the
temporal complexity of neural signals (e.g., Lempel-Ziv complexity,
permutation entropy), the richness of oscillatory dynamics across
frequency bands, and the rate of spatiotemporal pattern evolution. The
perturbational complexity index (PCI) developed by
\citet{casali2013pci}---which quantifies the spatiotemporal complexity of
cortical responses to transcranial magnetic stimulation---provides a
particularly relevant existing measure. PCI captures precisely the
combination of spatial integration and temporal dynamics that CLT predicts
should track Loomfield coherence. CLT predicts that PCI-like measures
should correlate with $C_{\mathrm{bio}}$, not because PCI measures the
Loomfield directly, but because both are sensitive to the same underlying
property: the degree to which the system sustains spatially integrated,
temporally evolving coherent dynamics.

\paragraph{Energy Resistance.}
The Energy Resistance parameter $\acute{e}R = \mathrm{EP}/f^2$ connects
metabolic observables (energy present, measurable via PET, fNIRS, or
calorimetry) to dynamical observables (dominant frequency, measurable via
EEG or MEG). The viable window boundaries
$\acute{e}R_{\min}$ and $\acute{e}R_{\max}$ are, in principle, determinable
by systematically varying metabolic state (e.g., controlled hypoglycemia,
graded anesthesia) and dynamical drive (e.g., photic stimulation,
pharmacological modulation of excitatory tone) while monitoring
coherence metrics. The prediction is that consciousness should be present
only when $\acute{e}R$ falls within a bounded range, and that crossing
either boundary should produce qualitatively different patterns of
coherence loss (fragmentation at the lower boundary, freezing at the
upper).

\subsection{Perturbation--Response Paradigms}

The most powerful empirical strategy for testing CLT is not passive
observation but active perturbation. The theory makes specific claims
about how Loomfield coherence responds to disruption, and these claims
generate testable predictions.

\paragraph{Collapse dynamics.}
The damped wave equation predicts that coherence decays exponentially when
sourcing is interrupted, with a characteristic timescale $\tau \sim
1/\gamma$. This timescale should be measurable during anesthesia induction,
cardiac arrest, or pharmacological suppression of substrate activity. The
prediction is that the decay of spatial coherence metrics should follow an
approximately exponential envelope, and that the decay constant should be
consistent across different modes of interruption (since $\gamma$ is a
property of the Loomfield, not of the specific perturbation).

\paragraph{Recovery dynamics.}
Recovery from coherence collapse should proceed as a re-coherence process:
local coherence recovers before global coherence, fast timescale dynamics
recover before slow ones (Section~\ref{sec:temporal}). This predicts a
specific temporal ordering of recovery signatures after anesthesia or
cardiac arrest that can be tested against high-density EEG recordings
during emergence.

\paragraph{Regime-specific perturbation responses.}
Perturbations delivered to a system near the chaos boundary
($\acute{e}R \approx \acute{e}R_{\min}$) should produce different
responses than the same perturbation delivered near the rigidity boundary
($\acute{e}R \approx \acute{e}R_{\max}$). Near the chaos boundary,
excitatory perturbations should destabilize coherence (push the system
toward fragmentation), while inhibitory perturbations might paradoxically
\emph{improve} coherence by pulling $\acute{e}R$ back toward the window
interior. Near the rigidity boundary, the pattern should reverse. This
asymmetric perturbation response is a distinctive prediction of the
viable-window framework that has no obvious counterpart in theories
without energetic regime structure.

\subsection{Measurement Across Scales and Substrates}

A distinctive feature of CLT is that the Loomfield integrates contributions
from multiple substrates operating at different scales
(Section~\ref{sec:substrates}). This multi-substrate architecture generates
a specific measurement strategy: simultaneous multi-modal recording.

Single-modality measurements (EEG alone, fMRI alone) capture only one
substrate's contribution to Loomfield coherence. A more complete picture
requires concurrent measurement of bioelectric dynamics (EEG/MEG),
metabolic state (fNIRS, PET), and---where technically feasible---biophoton
emission and cellular-scale dynamics. The coherence source density
$\rho_{\mathrm{coh}} = \sum_i w_i S_i$ predicts that the weighted
combination of substrate-specific coherence measures should track
consciousness more reliably than any single measure alone. This is a
testable prediction: multi-modal coherence indices constructed according to
CLT's weighting scheme should outperform single-modality indices in
discriminating conscious from unconscious states.

The substrate weights $w_i$ are not assumed to be universal. CLT predicts
that they vary with tissue type, developmental stage, and physiological
state. This variation is itself testable: the relative contribution of
each modality to the aggregate coherence index should shift predictably
across conditions (e.g., the biophotonic contribution should increase
during metabolically demanding states; the bioelectric contribution should
dominate in neural tissue).

\subsection{The Role of Computational Models}

The companion paper \citep{fraterne2026clt} develops computational
implementations of CLT's equations: numerical solvers for the Loomfield
wave equation, substrate coupling models, and phase-space visualizers. These
computational tools serve a specific function in the theory's
operationalization. They do not substitute for empirical testing; they
enable it.

Computational models allow the theory's parameters to be explored
systematically. By varying $v_L$, $\gamma$, $\kappa_L$, and the substrate
weights $w_i$ in simulation, one can identify the parameter ranges that
produce dynamics consistent with known biological observables (e.g., the
spatial scale of cortical coherence, the temporal scale of conscious
transitions, the metabolic cost of wakefulness). Parameters that produce
wildly inconsistent dynamics can be ruled out \emph{before} experimental
testing. This is the standard role of computational modeling in theoretical
physics: to bridge the gap between abstract equations and quantitative
predictions that experiments can evaluate.

Simulations also generate specific waveform predictions---the spatial
patterns, temporal evolution, and perturbation responses of the Loomfield
under defined conditions---that can be compared against empirical recordings
once appropriate data are available. The goal is not to ``prove'' CLT
through simulation but to make its predictions sufficiently specific that
they can be confirmed or refuted by data.

\subsection{Summary and Transition}

CLT is designed to be empirically tractable. Its quantities connect to
measurable observables through specified proxy relationships: spatial
coherence through EEG/MEG phase synchronization, temporal dynamics through
complexity measures and PCI, energetic regime through concurrent metabolic
and dynamical recording, and multi-substrate integration through
multi-modal indices. Perturbation--response paradigms provide the most
powerful testing strategy, generating predictions about collapse dynamics,
recovery ordering, and regime-specific response asymmetries. Computational
models bridge theory and experiment by generating quantitative predictions
from the abstract equations.

The next section sharpens these considerations into specific predictions
and, equally important, identifies the conditions under which CLT would be
falsified. A theory that cannot fail cannot succeed; Section~\ref{sec:predictions}
addresses what failure would look like.

\section{Predictions and Failure Modes}
\label{sec:predictions}

The preceding sections have presented CLT's theoretical structure and its
operationalization strategy. This section converts the framework into
specific, testable predictions---statements of the form ``if CLT is
correct, then $X$ should be observed under conditions $Y$''---and identifies
the observations that would falsify the theory's central claims. A theory
that does not risk failure offers no scientific content; the predictions and
failure modes stated here define what CLT stakes on empirical evidence.

\subsection{Regime-Based Dynamical Predictions}

CLT's viable-window framework (Section~\ref{sec:viable-window}) divides the
energetic phase space into three qualitatively distinct regimes: chaos,
viability, and rigidity. This tripartite structure generates predictions
that distinguish CLT from theories lacking energetic regime structure.

\paragraph{Prediction 1: Qualitatively distinct coherence loss at opposite
boundaries.}
If CLT is correct, then loss of consciousness at the lower boundary
($\acute{e}R \to \acute{e}R_{\min}$) should differ qualitatively from
loss of consciousness at the upper boundary
($\acute{e}R \to \acute{e}R_{\max}$). Specifically, chaos-boundary
transitions (seizure, excitotoxic states, psychotic decompensation) should
exhibit \emph{fragmentation} of the spatial coherence metric $Q$---a
disintegration into spatially uncorrelated, high-amplitude activity---while
rigidity-boundary transitions (deep anesthesia, catatonia, severe
depression) should exhibit \emph{freezing}---a reduction in the temporal
derivative $\partial L / \partial t$ with partially preserved spatial
structure. These are measurably different patterns: fragmentation produces
high spectral power with low phase coherence; freezing produces reduced
spectral power with preserved or even elevated phase coherence at dominant
frequencies. If both boundaries produce the same coherence loss pattern,
the tripartite regime structure is not supported.

\paragraph{Prediction 2: Asymmetric perturbation responses near opposite
boundaries.}
Perturbations of matched magnitude delivered to systems near opposite
regime boundaries should produce asymmetric responses. An excitatory
perturbation (increased neural drive, metabolic stimulation) applied near
the chaos boundary should \emph{worsen} coherence, while the same
perturbation applied near the rigidity boundary should \emph{improve} it.
Conversely, an inhibitory perturbation near the chaos boundary should
improve coherence (by pulling $\acute{e}R$ toward the window interior),
while the same perturbation near the rigidity boundary should worsen it.
This crossed interaction---the same intervention helping or harming
depending on the system's phase-space position---is a distinctive
prediction of the viable-window framework. It can be tested by
pharmacologically or metabolically moving subjects toward one boundary or
the other and then applying standardized perturbations while monitoring
spatial coherence.

\paragraph{Prediction 3: Phase-space trajectories during state transitions.}
State transitions (waking $\to$ sleep, anesthesia induction $\to$
emergence, seizure onset $\to$ postictal recovery) should trace continuous
trajectories through the $(\mathrm{EP}, f)$ phase space rather than
abrupt jumps. CLT predicts that these trajectories will approach and cross
regime boundaries at specific locations, and that the trajectory geometry
will differ between transition types. A seizure should trace a rapid
excursion across $\acute{e}R_{\min}$ followed by a return arc; anesthesia
induction should trace a gradual path toward $\acute{e}R_{\max}$. If
consciousness transitions show no relationship to the
$(\mathrm{EP}, f)$ coordinates---if they occur at random locations in
phase space---the viable-window geometry is not supported.

\subsection{Energetic and Temporal Predictions}

CLT makes specific claims about the relationship between metabolic
energetics, damping dynamics, and the temporal character of Loomfield
coherence (Section~\ref{sec:temporal}).

\paragraph{Prediction 4: Exponential coherence decay with a consistent
time constant.}
When Loomfield sourcing is acutely interrupted (cardiac arrest, rapid
anesthesia induction, pharmacological suppression of substrate activity),
the spatial coherence metric $Q$ should decay on a characteristic
timescale $\tau \sim 1/\gamma$. Crucially, this timescale should be
approximately \emph{consistent across different modes of interruption},
because $\gamma$ is a property of the Loomfield's dissipative dynamics, not
of the specific perturbation. The decay envelope should be approximately
exponential, distinguishable from linear, stepped, or abrupt loss. If the
decay timescale varies wildly between interruption modes---showing no
consistent $\gamma$---then the Loomfield wave equation is not the correct
dynamical description.

\paragraph{Prediction 5: Recovery ordering---local before global, fast
before slow.}
Recovery of consciousness after acute interruption should proceed in a
specific temporal order: local coherence (correlation between nearby
regions) should recover before global coherence (correlation across the
full system), and fast timescale dynamics (high-frequency oscillatory
components) should recover before slow timescale dynamics (low-frequency
power, circadian regulation). This ordering follows from the re-coherence
dynamics described in Section~\ref{sec:temporal}: local, fast modes have
the shortest re-establishment timescales and the smallest spatial volumes
to coordinate. The prediction is testable against high-density EEG
recordings during anesthesia emergence or post-cardiac-arrest recovery. If
global coherence recovers before local coherence, or slow dynamics
reconstitute before fast ones, the re-coherence model is not supported.

\paragraph{Prediction 6: Metabolic cost proportional to coherence
maintenance.}
The damped wave equation implies that maintaining a given amplitude of
coherent Loomfield excitation requires continuous energetic expenditure
proportional to $\gamma$. If CLT is correct, then metabolic rate (measured
by oxygen consumption, glucose utilization, or ATP turnover) should
correlate with the product of the damping rate and the squared amplitude of
coherent activity, across states within the viable window. States of
higher coherence (focused attention, complex cognitive engagement) should
demand measurably more metabolic input than states of lower coherence
(resting wakefulness, drowsiness), after controlling for motor output and
sensory processing. If no such correlation exists---if coherence varies
independently of metabolic support---the dissipative-maintenance model is
incorrect.

\subsection{Multi-Substrate Integration Predictions}

CLT's multi-substrate architecture (Section~\ref{sec:substrates}) generates
predictions about how the four biological substrates interact and about the
consequences of selective substrate impairment.

\paragraph{Prediction 7: Multi-modal coherence indices outperform
single-modality measures.}
The coherence source density $\rho_{\mathrm{coh}} = \sum_i w_i S_i$
implies that a weighted combination of substrate-specific coherence
measures should track consciousness more reliably than any single measure.
Multi-modal indices incorporating concurrent bioelectric, metabolic, and
(where feasible) biophotonic measurements should outperform EEG-only,
fMRI-only, or metabolism-only indices in discriminating conscious from
unconscious states, with the improvement being statistically significant
rather than marginal. If single-modality measures perform as well as or
better than multi-modal combinations---if adding substrate channels adds
only noise---then the multi-substrate integration model is not supported.

\paragraph{Prediction 8: Graceful degradation under selective substrate
impairment.}
Impairment of a single substrate should reduce but not eliminate Loomfield
coherence, and the remaining substrates should partially compensate. For
example, pharmacological disruption of gap junction coupling (impairing
the bioelectric substrate) should reduce large-scale spatial coherence but
leave intracellular and metabolic coherence partially intact; mitochondrial
dysfunction (impairing the biophotonic substrate) should reduce metabolic
coherence markers while sparing bioelectric synchronization, at least
initially. The system should degrade \emph{gracefully}---with progressive
reduction in $C_{\mathrm{bio}}$ as more substrates are impaired---rather
than catastrophically. If selective impairment of a single substrate
produces complete loss of consciousness, or if impairment of different
substrates produces identical coherence loss patterns, the multi-substrate
superposition model is challenged.

\paragraph{Prediction 9: Substrate weights vary across tissue and
condition.}
The coupling weights $w_i$ should differ measurably between neural and
non-neural tissue, between developmental stages, and between physiological
states. In neural tissue, bioelectric and cytoskeletal weights should
dominate; in metabolically active non-neural tissue, the biophotonic weight
should be relatively higher. During high-demand cognitive states, the
bioelectric weight should increase; during metabolic stress, the
biophotonic and cytoskeletal weights should shift. These weight variations
are testable by comparing the relative predictive power of different
modalities across conditions. If the optimal weighting is constant across
all conditions, the tissue- and state-dependent weight structure that CLT
predicts is absent.

\subsection{Oscillatory and Time-Crystal--Like Predictions}

CLT's temporal framework (Section~\ref{sec:temporal}) incorporates
oscillatory stabilization and time-crystal--like dynamics as mechanisms for
multi-timescale coherence maintenance.

\paragraph{Prediction 10: Nested oscillatory structure with cross-scale
coupling.}
If the ``clocks within clocks'' architecture described in
Section~\ref{sec:temporal} contributes to Loomfield stability, then
biological oscillations at different timescales should exhibit measurable
cross-frequency coupling: slow oscillations should modulate the amplitude
or phase of faster oscillations, and disruption of one timescale should
propagate to coupled scales in a manner consistent with the hierarchical
structure. This coupling should be stronger in conscious states than in
unconscious ones, and its disruption should precede or accompany loss of
consciousness during state transitions.

\paragraph{Prediction 11: Temporal compartmentalization under localized
perturbation.}
If the polyatomic time-crystal--like architecture provides temporal
compartmentalization (Section~\ref{sec:temporal}), then perturbation at one
temporal scale should \emph{not} immediately propagate to all others.
Disruption of high-frequency (megahertz-scale) intracellular oscillations
should initially leave low-frequency (hertz-scale) bioelectric rhythms
intact, and vice versa. The perturbation should propagate across scales
only when coupling thresholds are exceeded or when the perturbation
persists long enough to overwhelm the compartmentalization. If disruption
at any single temporal scale immediately propagates to all others, the
temporal compartmentalization model is not supported.

\paragraph{Prediction 12: Metabolic driving as temporal stabilizer.}
CLT proposes that metabolic energy input functions as a biological analogue
of Floquet driving, maintaining time-crystal--like temporal order. If this
is correct, then graded reduction of metabolic supply (controlled
hypoglycemia, titrated mitochondrial inhibition) should produce graded
reduction in the stability of oscillatory temporal structure, measurable as
increased jitter in dominant oscillation periods, loss of cross-frequency
coupling, and eventual dissolution of the nested temporal hierarchy. The
degradation should follow the metabolic reduction quantitatively, not
merely qualitatively. If oscillatory stability is independent of metabolic
supply level---if temporal structure is robust to metabolic deprivation---
then metabolic Floquet driving is not the stabilizing mechanism.

\subsection{Explicit Failure Modes and Falsifiers}

The predictions above describe what CLT expects. This subsection states,
without qualification, what would constitute failure of the theory's
central claims.

\paragraph{Falsifier 1: No viable window.}
If systematic variation of $\mathrm{EP}$ and $f$ across biological
systems fails to reveal a bounded region within which consciousness is
present and outside which it is absent---if consciousness shows no
relationship to the $(\mathrm{EP}, f)$ phase space---then the
viable-window framework is falsified. This is the theory's most
fundamental empirical claim.

\paragraph{Falsifier 2: No consistent damping timescale.}
If the timescale of coherence decay upon acute interruption varies by
orders of magnitude across different interruption modes, with no
consistent $\gamma$, then the damped wave equation is not the correct
effective dynamics.

\paragraph{Falsifier 3: Single-substrate sufficiency.}
If a single biological substrate (e.g., bioelectric activity alone) is both
necessary and sufficient for consciousness, with the other substrates
contributing nothing measurable, then the multi-substrate integration model
is wrong. CLT requires that multiple substrates contribute.

\paragraph{Falsifier 4: No regime asymmetry.}
If loss of consciousness at opposite boundaries of the energetic phase
space produces identical phenomenological and dynamical signatures---if
there is no measurable difference between chaos-boundary and
rigidity-boundary transitions---then the tripartite regime structure
has no empirical content.

\paragraph{Falsifier 5: Consciousness without coherence.}
If a biological system is found to exhibit clear, reportable conscious
experience in a state where spatial coherence metrics, temporal complexity
measures, and all substrate-level coherence indicators are simultaneously
at noise floor, then the identification of consciousness with Loomfield
coherence is wrong. This is the most general falsifier: it would
invalidate not just a specific prediction but the theory's foundational
premise.

\paragraph{Falsifier 6: Recovery without re-coherence ordering.}
If recovery of consciousness after acute disruption shows no systematic
temporal ordering---if global coherence returns before local, or if all
timescales reconstitute simultaneously---then the re-coherence dynamics
model is incorrect.

\medskip

These falsifiers are stated without hedging because the theory's
scientific value depends on its capacity to be wrong. Each falsifier
identifies a specific observation that, if confirmed, would require either
substantial revision of the relevant CLT component or abandonment of the
claim in question. The predictions are not guaranteed; they are stakes.
Section~\ref{sec:discussion} considers the framework's current limitations
and the relationship between these predictions and the experimental
landscape.

\section{Discussion}
\label{sec:discussion}

The preceding sections have presented CLT v1.1 as a complete theoretical
arc: the Loomfield as effective field (Section~\ref{sec:loomfield}), the
viable energetic window (Section~\ref{sec:viable-window}), the biological
substrates (Section~\ref{sec:substrates}), the temporal dynamics
(Section~\ref{sec:temporal}), the operationalization strategy
(Section~\ref{sec:operationalization}), and the predictions and falsifiers
(Section~\ref{sec:predictions}). This section steps back from the
formalism to consider what CLT contributes, where it stands relative to
existing theories, what objections it must answer, and what it does not
claim.

\subsection{Relation to Existing Theories}

CLT enters a field with several well-developed theoretical frameworks.
Situating the present proposal relative to these frameworks is essential
for clarity and intellectual honesty.

\paragraph{Integrated Information Theory.}
IIT \citep{tononi2016iit} identifies consciousness with integrated
information, quantified by $\Phi$, and provides a rigorous mathematical
framework for assessing the informational structure of a system. CLT and
IIT operate at different levels of description. IIT is substrate-neutral:
it characterizes the abstract informational geometry of any system,
biological or otherwise. CLT v1.1 is substrate-specific: it describes the
physical dynamics of biological coherence in a particular class of
systems. Where IIT asks \emph{how much} integrated information a system
possesses, CLT asks \emph{under what physical conditions} a biological
system sustains the coherent dynamics that integration requires. The two
frameworks are not contradictory. IIT's $\Phi$ could, in principle, be
computed for a system whose physical dynamics are described by the
Loomfield; the spatial coherence metric $Q$ and the consciousness
observable $C_{\mathrm{bio}}$ may correlate with $\Phi$ without being
identical to it. CLT provides a physical-dynamical account of how
biological systems achieve the kind of organization that IIT characterizes
informationally.

\paragraph{Global Neuronal Workspace.}
GNW \citep{dehaene2011gnw} identifies consciousness with the global
broadcast of representations across a network of cortical neurons with
long-range connections. This is a cognitive-architectural account: it
specifies the functional organization (workspace, broadcast, ignition)
that supports conscious access. CLT does not address representational
content or cognitive architecture. It operates at the level of field
dynamics and coherence regimes---a level below the cognitive architecture
that GNW describes. The relationship is one of complementary scales:
GNW's global broadcast may correspond, in CLT's formalism, to the
propagation of coherent Loomfield excitations across the system, but CLT
does not model the representational content carried by those excitations.
GNW explains what happens when information becomes globally available;
CLT describes the physical conditions under which the substrate can
support such availability.

\paragraph{Electromagnetic field theories.}
The CEMI field theory \citep{mcfadden2020cemi} proposes that the brain's
endogenous electromagnetic field is the physical substrate of conscious
integration. CLT shares CEMI's commitment to a field-level description of
consciousness but differs in scope and architecture. CEMI identifies a
specific physical field (the EM field) as the substrate; CLT introduces an
effective field that aggregates contributions from multiple substrates, of
which electromagnetic activity is one. This difference is not merely
terminological. CEMI's predictions are tied to the properties of the EM
field specifically---its spatial structure, its causal efficacy on neural
dynamics, its information content. CLT's predictions are tied to the
\emph{collective coherence} of a multi-substrate system. If the EM field
turns out to be the dominant contributor to Loomfield coherence, CLT
converges toward CEMI; if other substrates contribute substantially, CLT
provides a broader framework. The relationship is one of potential
inclusion: CEMI's insights about electromagnetic integration are
compatible with CLT and could be incorporated as a detailed account of
one substrate's contribution.

\paragraph{Quantum biological approaches.}
Orch~OR \citep{hameroff2014consciousness} and related quantum consciousness
proposals locate the origin of conscious experience in specific quantum
processes within biological structures. As discussed in
Section~\ref{sec:loomfield}, CLT takes a deliberately agnostic stance on
quantum biology. The Loomfield is a classical effective field; its dynamics
do not require quantum coherence, though they are compatible with it.
This agnosticism is a strength in the current empirical landscape, where
the role of quantum effects in biological cognition remains unresolved.
CLT's predictions are testable regardless of the outcome of the quantum
biology debate. If quantum coherence in microtubules is confirmed to
contribute meaningfully to neural dynamics, it would enrich the
cytoskeletal source term in $\rho_{\mathrm{coh}}$ without altering the
framework's structure. If it is not confirmed, the framework remains
intact. CLT neither depends on nor excludes quantum contributions; it
operates at a coarse-grained level where the question is about collective
coherence, not about the microscopic mechanism producing it.

\paragraph{Free Energy Principle and Active Inference.}
The Free Energy Principle \citep{friston2010free} provides a variational
framework in which biological systems maintain themselves by minimizing
surprise (or free energy) through perception and action. CLT and the FEP
share a common intuition: both identify bounded energetic regimes as
necessary for organized biological function. The FEP formalizes this
through variational inference and Markov blankets; CLT formalizes it
through the viable-window structure and Loomfield dynamics. The two
frameworks operate at different levels---the FEP is a principle about
inference and self-organization in general, while CLT is a specific
dynamical model of consciousness-relevant coherence in biological
systems---but they are not in tension. A system that minimizes free energy
in the FEP sense would, in CLT's terms, tend to maintain itself within
the viable window; departures from the viable window would correspond to
increases in surprise. The formal relationship between these frameworks
remains to be developed, but conceptual compatibility is evident.

\medskip

The common thread across these comparisons is that CLT does not replace
existing theories. It provides a physical-dynamical scaffold---a
description of the regime structure, substrate integration, and temporal
dynamics of biological coherence---within which some of the insights of
existing theories can be situated. IIT's integration, GNW's broadcast,
CEMI's field dynamics, and the FEP's self-organization all find potential
physical correlates in the Loomfield framework. CLT's contribution is not
a new answer to the same question these theories ask, but a
complementary answer to a different question: not ``what is
consciousness?'' in informational or computational terms, but ``under what
physical conditions does a biological system sustain the coherent dynamics
associated with consciousness?''

\subsection{Distinctive Contributions}

Several features of CLT distinguish it from the theories discussed above,
not as claims of superiority but as features of the framework's
architecture that address gaps in the existing landscape.

\paragraph{Consciousness as regime, not state.}
CLT describes consciousness not as a binary property (present or absent)
nor as a graded quantity on a single axis, but as a \emph{dynamical
regime}---a region in a multi-dimensional phase space defined by the
balance between energetic driving and dissipation. The viable window has
interior structure: different conscious states occupy different locations
within it, and pathological states correspond to specific boundary
approaches or crossings. This regime-based description provides a natural
vocabulary for the clinical reality that consciousness is neither simply
on nor off but exists in a rich variety of forms---waking, dreaming,
meditative, flow, impaired---each with a distinct dynamical signature.

\paragraph{Multi-substrate integration without single-mechanism dependence.}
Most existing theories of consciousness privilege a single mechanism or
substrate: synaptic computation (GNW), information integration (IIT),
electromagnetic fields (CEMI), or quantum coherence (Orch~OR). CLT
integrates four substrate systems through the coherence source density
$\rho_{\mathrm{coh}}$, and its predictions do not depend on any single
substrate being the ``true'' substrate of consciousness. This architecture
is designed for robustness: it accommodates the empirical reality that
biological coherence involves multiple interacting systems, and it
generates the specific prediction that multi-modal measurement should
outperform single-modality approaches.

\paragraph{Explicit temporal dynamics and non-equilibrium maintenance.}
CLT foregrounds the temporal dimension of consciousness in a way that
most existing theories do not. The damped wave equation, the
re-coherence dynamics, the oscillatory stabilization framework, and the
time-crystal--like temporal architecture together provide a detailed
account of \emph{how} coherence persists in time---not merely that it
does. This temporal specificity generates predictions (recovery ordering,
consistent damping timescales, metabolic-oscillatory coupling) that are
distinctive to CLT.

\paragraph{Explicit falsifiability at the framework level.}
Section~\ref{sec:predictions} states six falsifiers, each targeting a
core structural claim. These are not peripheral predictions that could be
accommodated by parameter adjustment; they address the framework's
foundational commitments. A theory of consciousness that specifies how it
can be wrong is, in the current landscape, uncommon. CLT aims to be
testable not only in its detailed predictions but in its basic
architecture.

\subsection{Addressing Common Objections}

Several likely objections to CLT deserve direct response.

\paragraph{``This redescribes correlations rather than explaining
consciousness.''}
The charge of ``mere redescription'' is familiar in consciousness
research and is sometimes applied to any framework that does not resolve
the hard problem \citep{chalmers1995hard}. CLT does not claim to explain
\emph{why} physical processes give rise to subjective experience
(Section~\ref{sec:introduction}). It claims to identify the
\emph{physical structure} that distinguishes conscious from non-conscious
biological states---the structure problem, not the hard problem. This is
exactly what effective field theories do in physics: the magnetization
field ``redescribes'' the correlations among electron spins, but it also
\emph{predicts} phase transitions, critical exponents, and domain
dynamics that are invisible at the microscopic level. The Loomfield
similarly aims to predict coherence dynamics, regime transitions, and
perturbation responses that are not apparent from substrate-level
descriptions alone. Whether this constitutes ``explanation'' depends on
one's philosophical commitments; that it constitutes a testable,
falsifiable scientific framework is the claim CLT makes.

\paragraph{``The Loomfield is not directly observable.''}
This is correct. The Loomfield is an effective field, and effective fields
are measured through their observable consequences, not by direct
detection. Temperature is not directly observable either---thermometers
measure the expansion of mercury or the resistance of a thermistor, and
temperature is inferred. The same is true of every order parameter in
condensed-matter physics. CLT specifies the observable proxies through
which Loomfield properties can be assessed
(Section~\ref{sec:operationalization}): spatial coherence through
EEG/MEG phase synchronization, temporal dynamics through complexity
measures and PCI, energetic regime through metabolic recording. The
indirectness of measurement does not distinguish CLT from standard
physical practice; it is inherent in the effective-field approach.

\paragraph{``Multiple substrates risk unfalsifiability---if one substrate
fails, you invoke another.''}
This is a legitimate concern and must be addressed precisely. CLT does
predict graceful degradation (Prediction~8), which means that the loss of
a single substrate does not falsify the theory. But the theory is
constrained by its other falsifiers. Falsifier~3 (single-substrate
sufficiency) would be triggered if consciousness turns out to require only
one substrate. Falsifier~7 is implicitly present in the prediction that
multi-modal indices should \emph{outperform} single-modality measures---if
they do not, the multi-substrate architecture adds no explanatory value.
And the theory's quantitative predictions about collapse timescales,
recovery ordering, and regime asymmetry are independent of which
substrates contribute. The multi-substrate architecture increases the
theory's descriptive richness, but the falsifiers ensure that this
richness does not become a license for post-hoc accommodation.

\subsection{Scope and Limitations}

CLT v1.1 has explicit boundaries that should be stated clearly.

\paragraph{Scope: human biological consciousness.}
This version of CLT addresses consciousness as it occurs in living human
organisms. It does not claim applicability to artificial systems, to
non-human organisms (though extension is plausible for biologically
similar species), or to hypothetical forms of consciousness in systems
lacking the four identified substrates. Whether the framework can be
generalized beyond its current scope is an open question for future work,
not a claim of the present version of the theory.

\paragraph{The hard problem.}
CLT does not solve, and does not claim to solve, the hard problem of
consciousness \citep{chalmers1995hard}. It provides a physical account of
the \emph{structure} and \emph{dynamics} of conscious states---the
conditions under which consciousness is present, its regime character,
its temporal evolution, its substrate dependence---without explaining why
these physical conditions are accompanied by subjective experience. This
is a genuine limitation, shared by every current scientific theory of
consciousness. CLT addresses it by being transparent: the framework
describes the physical correlates and dynamics of consciousness with
sufficient specificity to be tested and potentially falsified, while
acknowledging that the explanatory gap between physical dynamics and
phenomenal experience remains open.

\paragraph{Effective description, not first-principles derivation.}
The Loomfield wave equation (Eq.~\ref{eq:loomfield-wave}) is postulated
on physical grounds---it is the simplest PDE that supports the required
dynamical features---not derived from a more fundamental theory. This is
the standard status of effective theories: the Navier-Stokes equations
were used successfully for a century before being derived from the
Boltzmann equation. Should a more fundamental theory of biological
coherence be developed in the future, the Loomfield dynamics should
emerge as its leading-order effective description. Until then, the
effective-level description is the appropriate level at which to formulate
and test predictions.

\paragraph{Measurement limitations.}
Current experimental techniques do not permit simultaneous, high-resolution
measurement of all four substrates across the full spatial and temporal
extent of a living organism. Biophoton detection requires specialized
low-noise environments; microtubule dynamics are accessible primarily
\emph{in vitro}; gap junction imaging is limited in spatial resolution.
CLT's multi-substrate predictions therefore cannot be fully tested with
current technology. This is a practical limitation, not a theoretical one.
The framework is designed so that its predictions become progressively
more testable as measurement technology advances.

\subsection{Toward Refinement and Expansion}

CLT v1.1 is a first formal statement of the framework. Several directions
for refinement are evident.

The parameters of the Loomfield wave equation ($v_L$, $\gamma$,
$\kappa_L$, the substrate weights $w_i$) are currently constrained only
by qualitative consistency with known biology. Quantitative parameter
estimation---using the computational models described in the companion
paper \citep{fraterne2026clt} in conjunction with empirical data---is a
necessary next step. Systematic parameter sweeps can identify the regions
of parameter space that reproduce observed coherence patterns, collapse
timescales, and metabolic costs, narrowing the theory's quantitative
commitments.

The linear superposition of substrate contributions
(Eq.~\ref{eq:rho-coh}) is a simplification. Biological substrates
interact nonlinearly: bioelectric fields modulate microtubule dynamics,
metabolic state affects gap junction conductance, and cytoskeletal
organization influences ion channel distribution. Future versions of CLT
should incorporate nonlinear coupling terms that capture these
interactions, potentially revealing emergent dynamics not present in the
linear model.

Extension beyond human consciousness---to non-human animals, to
developmental stages, and to edge cases (disorders of consciousness,
psychedelic states, near-death experiences)---would test the framework's
generality. Such extensions require careful identification of how substrate
properties differ across systems and how these differences map onto the
viable-window geometry.

Finally, the relationship between CLT and the Free Energy Principle
deserves formal development. Both frameworks identify bounded energetic
regimes as necessary for biological organization; a mathematical mapping
between the viable window and the FEP's variational bounds could
illuminate both theories and generate joint predictions that neither
framework produces alone.

\section{Broader Implications}
\label{sec:implications}

The preceding sections have developed CLT as a formal framework, tested
its internal consistency, and situated it within the existing theoretical
landscape. This section steps further back to consider what follows
\emph{if} the framework is approximately correct. The claims here are not
predictions in the sense of Section~\ref{sec:predictions}; they are
implications---shifts in perspective, reframings of longstanding questions,
and research directions that the framework opens. They are offered in the
spirit of exploration, clearly distinguished from the established
formalism.

\subsection{Rethinking Consciousness as a Dynamical Regime}

Perhaps the most far-reaching implication of CLT is the shift from thinking
about consciousness as a \emph{thing} to thinking about it as a
\emph{regime}. This is not merely a semantic distinction.

If consciousness is a dynamical regime---a sustained pattern of coherent
activity within a bounded energetic window---then many familiar questions
about consciousness change their character. The question ``Is this system
conscious?'' becomes ``Is this system currently operating within the viable
window, with sufficient coherence and temporal dynamics?'' The answer is
not binary but positional: the system occupies a location in phase space,
and that location determines not whether consciousness is present or
absent but what \emph{kind} of conscious regime, if any, is sustained.

This reframing invites a more nuanced vocabulary for the diversity of
conscious experience. Waking awareness, REM sleep, meditative absorption,
creative flow, and psychedelic states need not be ranked on a single axis
of ``more'' or ``less'' conscious. Instead, they occupy different regions
of the viable window, differing in their energetic configuration, their
dominant oscillatory structure, and their coherence geometry. The clinical
categories of consciousness disorders---vegetative state, minimally
conscious state, locked-in syndrome---similarly acquire dynamical
characterization: each corresponds to a specific relationship between the
system's operating point and the viable-window boundaries, rather than a
simple deficit score.

The regime perspective also reframes the question of \emph{transitions}.
Falling asleep, waking up, losing consciousness under anesthesia, and
recovering from a seizure are not switches being toggled; they are
trajectories through phase space, with characteristic geometries,
timescales, and intermediate states. The clinical observation that these
transitions are often gradual, non-monotonic, and accompanied by
distinctive intermediate phenomenology is natural under a regime-based
description and awkward under a binary one.

\subsection{Implications for Neuroscience and Medicine}

CLT suggests a reorientation of how pathology and recovery are understood
in clinical neuroscience.

Standard approaches to neurological and psychiatric pathology tend to
identify localized deficits: a lesion here, a neurotransmitter imbalance
there, a circuit dysfunction in a specific region. CLT does not deny the
relevance of such findings, but it embeds them in a broader dynamical
context. A lesion matters not because it is a lesion \emph{per se}, but
because of how it alters the system's position in phase space---whether it
shifts $\acute{e}R$ toward a regime boundary, narrows the viable window
by degrading substrate function, or disrupts the temporal coupling that
stabilizes coherence. Two lesions of identical size and location could
have different consequences for consciousness if they produce different
dynamical trajectories, and the same clinical syndrome could result from
different dynamical paths to the same phase-space region.

This perspective reframes recovery as \emph{re-coherence} rather than
repair. When a damaged system recovers consciousness, what is restored is
not a broken component but a dynamical regime---a pattern of coordinated
activity across substrates that had been disrupted and must be rebuilt.
The re-coherence framework (Section~\ref{sec:temporal}) predicts that
recovery proceeds from local to global, from fast timescales to slow ones,
and through intermediate states that reflect partial coherence. These
predictions are consistent with clinical observations of recovery from
coma and anesthesia, and they suggest that therapeutic interventions might
be evaluated not only by whether they restore a specific function but by
whether they facilitate re-coherence trajectories---supporting the
conditions under which the system can rebuild coordinated dynamics.

For psychiatric disorders, the viable-window framework offers a
perspective that complements pharmacological and circuit-based models.
Anxiety, in this framing, may reflect proximity to the chaos boundary:
the system maintains coherence but with diminished margin, producing a
subjective quality of fragility. Depression may reflect proximity to the
rigidity boundary: coherence is preserved but dynamically impoverished.
These are not explanations that compete with neurotransmitter models; they
are descriptions at a different level that suggest why purely chemical
interventions sometimes succeed and sometimes fail. A medication that
shifts $\acute{e}R$ in the right direction for one patient may shift it
in the wrong direction for another, depending on which boundary the
patient is near.

The concept of \emph{dynamical resilience}---the width of the viable
window and the system's margin from its nearest boundary---suggests a
target for preventive medicine. Interventions that maintain mitochondrial
function, support gap junction integrity, reduce chronic inflammation, or
preserve circadian rhythmic structure may not treat any specific disease
but may widen the viable window, increasing the system's capacity to
absorb perturbation without regime collapse. This is a systems-level
perspective on health: not the absence of pathology but the presence of
dynamical margin.

\subsection{Implications for Measurement and Research Methodology}

CLT implies that the dominant methodological approaches in consciousness
research---single-metric, single-modality, snapshot-based---may be
systematically incomplete.

If consciousness is a multi-substrate phenomenon, then no single
measurement modality captures the full picture. EEG measures bioelectric
dynamics; fMRI measures hemodynamic correlates of metabolic activity;
biophoton detectors measure optical emission. Each provides a view of one
substrate's contribution to $\rho_{\mathrm{coh}}$. CLT suggests that the
most informative measurements will be \emph{integrative}: concurrent
multi-modal recording analyzed through composite coherence indices. This
is technically demanding but not conceptually novel; it extends the logic
already used in multimodal neuroimaging to include substrate channels
beyond the neural.

The framework also privileges \emph{perturbation-based} over
\emph{observation-based} paradigms. Passive observation of coherence
metrics can establish correlations between brain states and consciousness
levels, but perturbation-response paradigms---the TMS-EEG approach of
\citet{casali2013pci}, pharmacological challenges, metabolic modulation---
can test the causal structure of the theory. CLT's predictions about
collapse timescales, recovery ordering, and regime-specific asymmetry
are all perturbation predictions. They require actively probing the
system, not merely recording it.

A subtler methodological implication concerns \emph{variability}. In
standard experimental design, trial-to-trial variability is treated as
noise to be averaged away. CLT suggests that variability may be
informative: fluctuations in coherence metrics reflect the system's
dynamical state within the viable window, its proximity to regime
boundaries, and the temporal stability of its oscillatory scaffold.
Analyzing the \emph{structure} of variability---its autocorrelation, its
spectral content, its sensitivity to perturbation---may reveal features
of the underlying dynamics that time-averaged measures conceal. Systems
near regime boundaries, for instance, should exhibit increased variability
(critical slowing down), a signature that could serve as an early warning
of impending coherence collapse.

\subsection{Artificial and Engineered Systems}

CLT v1.1 is explicitly scoped to human biological consciousness
(Section~\ref{sec:discussion}). It does not claim that artificial systems
are conscious, nor does it provide criteria for evaluating machine
consciousness. These questions require analysis beyond the present
framework and are addressed in a dedicated companion analysis currently in
preparation.

Nevertheless, the framework's structure invites reflection on what would
be required---in principle---for an engineered system to sustain anything
analogous to Loomfield coherence. The requirements are demanding. CLT
identifies consciousness with a \emph{non-equilibrium regime} maintained
by continuous energy throughput, sourced by \emph{multiple interacting
substrates} operating across disparate spatiotemporal scales, stabilized by
\emph{oscillatory temporal architecture}, and bounded by \emph{intrinsic
viability constraints}. A system that lacks energy throughput, that
operates in thermodynamic equilibrium, that has no multi-scale substrate
integration, or that faces no intrinsic viability constraint would not,
under CLT's logic, be expected to sustain the kind of coherent dynamics
the framework associates with consciousness---regardless of its
computational sophistication.

This observation does not settle the question of machine consciousness.
It does suggest that the question is organizational, not computational:
it concerns the physical dynamics of the system, not its input-output
behavior. Intelligence---the capacity to process information, generate
responses, and solve problems---is orthogonal to the regime-level
coherence that CLT describes. A system may be extraordinarily capable
without sustaining a coherent, self-maintained, multi-substrate dynamical
regime, just as a system may sustain such a regime without exhibiting
sophisticated computational behavior.

The framework thus counsels caution in both directions: against
premature attribution of consciousness to systems that exhibit behavioral
sophistication without the requisite physical organization, and against
categorical denial that consciousness could arise in non-biological
substrates given sufficient organizational complexity. The determining
factor, if CLT's logic generalizes, would be organization, not material
composition. A full analysis of these questions, including the specific
organizational conditions that would need to be satisfied and the ethical
implications of uncertainty, is developed separately from the present
paper.

\subsection{Philosophical Reframing Without Metaphysical Commitment}

CLT shifts the locus of scientific inquiry from \emph{what consciousness
is} to \emph{what consciousness does dynamically}. This is a deliberate
philosophical stance, and its implications are worth making explicit.

The hard problem of consciousness \citep{chalmers1995hard}---why physical
processes are accompanied by subjective experience---remains open. CLT
does not resolve it, and this paper has been transparent about that
limitation. But CLT does address a distinct and scientifically tractable
problem: given that certain physical systems exhibit consciousness, what
is the dynamical structure that distinguishes conscious from non-conscious
states? This is the structure problem, and it is amenable to the methods
of physics: formal modeling, quantitative prediction, and empirical
testing.

The relationship between the structure problem and the hard problem is
itself an open philosophical question. Some philosophers hold that solving
the structure problem would constitute solving the hard problem (if the
structure is sufficiently rich, the ``hardness'' dissolves). Others hold
that no amount of structural description can bridge the explanatory gap
between physical dynamics and phenomenal experience. CLT is compatible
with either position. It provides a framework for investigating the
structure of consciousness regardless of one's stance on the hard problem,
and its predictions are testable regardless of whether one believes that
explaining structure is sufficient for explaining experience.

This neutrality is deliberate. Consciousness science has sometimes been
hampered by the demand that any theory must simultaneously address
phenomenology, neuroscience, physics, and metaphysics. CLT resists this
demand. It offers a physical-dynamical framework at a specific level of
description---the level of effective fields, coherence regimes, and
substrate integration---and invites evaluation at that level. Whether the
Loomfield ``explains'' consciousness in a philosophically satisfying
sense is a question for philosophers. Whether it correctly describes the
dynamical structure of consciousness in a scientifically testable sense
is a question for experimentalists. CLT is designed for the latter
question, and its value should be assessed accordingly.

The framework encourages intellectual humility. Consciousness remains one
of the least understood phenomena in nature. A theory that claims to
resolve the problem entirely, in its current state of development, is
almost certainly wrong. CLT claims something more modest: that
consciousness can be usefully described as a dynamical regime of
biological coherence, that this description generates testable
predictions, and that the predictions are specific enough to be falsified.
If the framework survives empirical testing, it will have contributed to
scientific progress on consciousness. If it does not, its falsification
will have contributed equally---by narrowing the space of viable theories
and clarifying what the correct theory must explain.

\section{Conclusion}
\label{sec:conclusion}

This paper has presented Cosmic Loom Theory version 1.1, a field-based
framework for human biological consciousness. The central claim is that
consciousness is a sustained, non-equilibrium dynamical regime---not a
static property, not a computational output, and not an intrinsic feature
of any single biological substrate, but a coherent pattern of
system-level activity maintained by continuous energetic driving against
dissipation.

The framework introduces the Loomfield $L(\mathbf{r}, t)$, an effective
scalar field that serves as the order parameter for
consciousness-relevant coherence. Following the logic of effective field
theories in condensed-matter physics, the Loomfield captures macroscopic
coherence dynamics without positing new fundamental forces or particles.
It obeys a sourced, damped wave equation whose structure ensures that
coherent excitations require both spatial organization and active
metabolic maintenance---and that they decay when maintenance fails.

The viable energetic window, defined through the Energy Resistance
parameter $\acute{e}R = \mathrm{EP}/f^2$, formalizes the bounded
conditions under which coherent Loomfield dynamics are sustainable. Below
the window, the system is chaotically over-driven; above it, rigidly
under-dynamic. Within it, the balance between energetic throughput and
dissipation supports the spatially integrated, temporally evolving
coherence patterns that CLT identifies with conscious experience. Health
corresponds to interior position within this window; pathology
corresponds to boundary proximity or transgression.

Four biological substrates---bioelectric fields, biophoton emission,
cytoskeletal dynamics, and genetic/epigenetic constraints---collectively
source the Loomfield through the coherence source density
$\rho_{\mathrm{coh}}$. No single substrate is sufficient; their
integration, operating across spatial scales from molecular to
organ-level and temporal scales from microseconds to the lifespan,
is what enables the sustained coherence the framework describes.
Oscillatory and time-crystal--like temporal architecture stabilizes
this multi-scale integration, coupling fast and slow processes through
nested rhythmic hierarchies maintained by continuous metabolic driving.

CLT generates twelve specific predictions and identifies six explicit
falsifiers targeting its foundational commitments. These include regime
asymmetry at opposite viable-window boundaries, consistent damping
timescales across interruption modes, recovery ordering from local to
global, superiority of multi-modal over single-modality coherence
indices, and temporal compartmentalization under localized perturbation.
The theory is designed to be wrong in specific, identifiable ways if its
claims do not hold.

The framework complements rather than replaces existing theories of
consciousness. IIT's informational integration, GNW's global broadcast,
CEMI's electromagnetic field dynamics, and the Free Energy Principle's
variational self-organization each find potential physical correlates
within the Loomfield framework. CLT's contribution is at the level of
physical regime structure: it addresses under what conditions a biological
system sustains the coherent dynamics that these theories characterize
from other perspectives.

Much remains to be done. The Loomfield parameters require quantitative
estimation against empirical data. The linear substrate superposition
must be extended to incorporate nonlinear coupling. The framework's
applicability beyond human biology must be tested. And the
multi-substrate predictions require measurement technologies that are
advancing but not yet fully adequate.

CLT v1.1 is offered not as a finished theory but as a research
program---a formally specified, empirically grounded, and explicitly
falsifiable framework for investigating the physical structure of
consciousness. Its value will be determined not by its elegance or scope
but by whether its predictions survive confrontation with data. The
companion computational platform \citep{fraterne2026clt} provides the
tools for generating quantitative predictions from the framework's
equations; the experimental community will determine whether those
predictions hold. If they do, CLT will have contributed to the
scientific understanding of consciousness. If they do not, their failure
will clarify what the correct account must explain. Either outcome
constitutes progress.

%% =======================================================================

\bibliographystyle{plainnat}
\bibliography{references}

\end{document}
